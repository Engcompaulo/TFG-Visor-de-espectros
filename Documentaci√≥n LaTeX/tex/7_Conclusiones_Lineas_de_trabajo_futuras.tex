\capitulo{7}{Conclusiones y Líneas de trabajo futuras}

En esta sección se exponen las conclusiones obtenidas al terminar el desarrollo
del proyecto y se comentan anotaciones que podrían seguirse para avanzar con
este trabajo en el futuro.

\section{Conclusiones}

A una semana de la entrega y habiendo terminado el desarrollo del proyecto puedo
considerar que se han cumplido los objetivos definidos y el producto resultante
es una aplicación que va a facilitar el trabajo de Susana, la geóloga
colaboradora, además de una herramienta de minería de datos que puede llegar a
ser muy útil para gente que sin muchos conocimientos de minería de datos pueda
usar estas técnicas sin problema.

También puedo afirmar que este proyecto ha sido bastante ambicioso respecto a
todo lo que se quería que ofreciese, provocando que se centrase demasiado en
añadir todas las funcionalidades y descuidando, por desgracia, algunas
cuestiones referentes a seguridad, tolerancia a fallos, diseño y pruebas del
sistema, cuestiones planteadas como líneas futuras. Aún así estoy muy contento
del producto final y de todo lo que ofrece.

Como la etapa final de aprendizaje del grado destacar la cantidad enorme de
conocimientos nuevos que se adquieren. En mi caso este conocimiento se centra
principalmente en desarrollo web, \textit{front-end} y \textit{back-end},
conocimientos que era de obligatorio cumplimiento su adquisión. Me parece que
estos temas deberían explorarse durante el grado por, en mi opinión, ser unos
conocimientos básicos que todo graduado en informática debería poseer.

Otros conocimientos adquiridos han sido el manejo de bases de datos NoSQL, en
concreto bases documentales, un tema sobre el que no se esperaba aprender pero
que seguro es de gran utilidad en el futuro, personal y profesionalmente.

El último punto que quiero comentar, y de los que más ilusión me ha traído
durante el desarrollo es ver, desarrollar y aplicar técnicas de minería de datos
a ejemplos reales de investigación. Aunque día a día salgan noticias sobre
avances en inteligencia artificial parece algo lejano, a veces casi de ciencia
ficción. Pero ser capaz de aplicar estas técnicas, aprender cómo funcionan y,
sobre todo, poder decir frases como ``El proyecto que he desarrollado es capaz
de decirte la profundidad a la que se ha extraído un mineral'' consigue hacer de
esta disciplina algo más cercano.

\section{Líneas de trabajo futuras}

Este trabajo esta pensado para evolucionar en el futuro hacia un proyecto más
ambicioso de tal forma que sea posible su uso en otros ámbitos aparte de la
geología. A continuación se presenta una lista de tareas a realizar para
continuar con su desarrollo:
\begin{itemize}
	\item Mejorar la seguridad y tolerancia a fallos en las búsquedas a la base de
	datos.
	\item Mejorar el diseño general de los modelos.
	\item Modificar la aplicación de forma que sea capaz de trabajar con atributos
	definidos por el usuario, en vez de que estos estén fijos.
	\item Mejorar la creación de clasificadores para poder elegir sobre que
	atributo crearlos, en vez de todos a la vez.
	\item Poder aplicar un preprocesamiento de los datos antes de crear el
	clasificador en vez de usar uno fijo y por defecto.
	\item Crear unos tests más exhaustivos.
	\item Mejorar el sistema de usuarios añadiendo opciones como la eliminación de
	la cuenta.
	\item Poder poner tareas que requieran de más tiempo en segundo plano para que
	el usuario pueda seguir usando la aplicación.
	\item Teniendo en cuenta el punto anterior, añadir un sistema de
	notificaciones por correo electrónico para avisar al usuario que las tareas que
	estaban en segundo plano han terminado.
\end{itemize}

