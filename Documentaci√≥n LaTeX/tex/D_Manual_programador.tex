\apendice{Manual del programador}

\section{Introducción}

\section{Estructura de directorios}

\section{Manual del programador}

\subsection{Manual de despliegue}
El despliegue de la aplicación se hace mediante la herramienta Nanobox, cuyo
propósito principal es el de facilitar la tarea de despliegue. Para ello combina
servicios de virtualización con servidores virtuales privados o VPS, ocupándose
de la configuración de la máquina virtual que el proveedor de VPS nos
proporcione.\\

El primer paso es registrarse en alguno de los proveedores
disponibles\footnote{\url{https://docs.nanobox.io/providers/hosting-accounts/}}.
Para este proyecto se ha elegido Digital Ocean debido a que durante el proyecto
se contaba con el ``Student Developer Pack'' de GitHub, el cual contiene un
crédito gratuito de 50\$ para esa plataforma.\\

El siguiente paso es crear una cuenta en
Nanobox\footnote{\url{https://nanobox.io/}}. Una vez hecho, hay que enlazar esta
cuenta con la creada anteriormente, desde las opciones de la cuenta en la
pestaña ``Hosting Accounts'', una vez ahí seguir las instrucciones que se
muestran en la página.\\

Después de tener las cuentas preparadas hay que descargar la herramienta Nanobox
para el sistema adecuado e instalarla, seguir las instrucciones de la
documentación oficial\footnote{\url{https://docs.nanobox.io/install/}} y las del
instalador.\\

Desde la página principal se pulsa en ``Lauch New App'' para crear la nueva
aplicación, seguir las instrucciones en la página. Al terminar se muestran
instrucciones para desplegar la aplicación.\\

Como se puede comprobar, Nanobox cumple el propósito de facilitar el despliegue.
Para cualquier otra duda sobre la herramienta toda la documentación oficial está
disponible en \url{https://docs.nanobox.io/}.\\
\section{Compilación, instalación y ejecución del proyecto}

\section{Pruebas del sistema}

