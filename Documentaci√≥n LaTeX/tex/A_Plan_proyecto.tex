\apendice{Plan de Proyecto Software}

\section{Introducción}
Para que un proyecto se desarrolle con normalidad y con el menor número de
imprevistos posibles es esencial que cuente con una fase de planificación. Aquí
se estima el tiempo, trabajo y dinero que puede llegar a usarse durante la
realización del proyecto. Para ello, se debe analizar en detalle cada parte del
proyecto. De cara al futuro, el análisis del proyecto puede servir para predecir
como de bien puede desarrollarse una continuación del mismo.

La planificación del proyecto consta de dos partes:
\begin{itemize}
	\tightlist
	\item \textbf{Planificación temporal}: en esta parte se analiza y planifica el
	tiempo que se va a dedicar a cada parte del proyecto, fecha de inicio y final
	aproximado, teniendo en cuenta el trabajo necesario para cada parte.
	\item \textbf{Estudio de viabilidad}: en esta parte se analiza como de viable
	es la realización del proyecto, se divide a su vez en dos apartados:

	\begin{itemize}
		\tightlist
		\item \textbf{Económica}: en esta parte se estiman los costes y los beneficios
		que puede suponer el proyecto.
		\item \textbf{Legal}: en esta parte se analizan los conceptos legales del
		proyecto, como podrían ser las licencias del proyecto o la política de
		protección de datos.
	\end{itemize}
\end{itemize}

\section{Planificación temporal}
La planificación temporal se organiza mediante \textit{sprints}. Cada
\textit{sprint} dura una o dos semanas. Al terminar cada \textit{sprint} se
realiza una reunión con los tutores para dar por terminado

\subsection{Sprint 0}
En este \textit{sprint} marcó el inicio del proyecto. La lista de tareas está
disponible en
\href{https://github.com/IvanBeke/TFG-Visor-de-espectros/milestone/1?closed=1}{Sprint 0}\footnote{\url{https://github.com/IvanBeke/TFG-Visor-de-espectros/milestone/1?closed=1}}.
En reuniones previas se habló con los tutores en que iba a consistir en
proyecto, pero no estaba claro con que tecnologías desarrollarlo. Se decidió
hacer una evaluación de las tecnologías posibles y crear unos prototipos
básicos. Todas las tareas se completaron a tiempo.\\

\subsection{Sprint 1}
En este \textit{sprint} se habló sobre el despliegue de la aplicación. Los
tutores comentaron que en proyectos webs anteriores el despliegue se solía dejar
para las etapas finales del proyecto, haciendo que todos los problemas asociados
surjan en esas finales, retrasando el despliegue y, a veces, no llegar a
desplegar la aplicación.\\

Se conocía la plataforma Heroku para ello y se probó a usarla, adicionalmente se
buscaron otras alternativas. También se usó el prototipo escogido para crear el
proyecto definitivo y se trabajó en la memoria. La última parte fue estudiar las
guías de estilo de Python, aplicarlas en los prototipos y documentar su
código.\\

La lista de tareas está disponible en
\href{https://github.com/IvanBeke/TFG-Visor-de-espectros/milestone/2?closed=1}{Sprint 1}\footnote{\url{https://github.com/IvanBeke/TFG-Visor-de-espectros/milestone/2?closed=1}}.
Todas las tareas se completaron a tiempo.\\

\subsection{Sprint 2}
Dado que en la aplicación se necesita que los usuarios suban contenido, se
necesita control de usuarios, en este \textit{sprint} se investigaron formas de
ofrecerlo. También debido a la necesidad de disponer almacenamiento persistente
se tuvo que mirar otras formas de despliegue y buscar como Heroku lo ofrece, que
de serie no lo hace. Al final se decidió cambiar a Nanobox.\\

La lista de tareas está disponible en
\href{https://github.com/IvanBeke/TFG-Visor-de-espectros/milestone/3?closed=1}{Sprint 2}\footnote{\url{https://github.com/IvanBeke/TFG-Visor-de-espectros/milestone/3?closed=1}}.
Todas las tareas se completaron a tiempo.\\

\subsection{Sprint 3}
En este \textit{sprint} se podría decir que comienza el desarrollo del proyecto,
basado en el prototipo. Como tal se mejoró el aspecto visual de la aplicación,
se cambio la estructura para tener partes diferenciadas y mantenibles, añadir
control de usuarios y mejorar la subida de ficheros.\\

También se planteó añadir subir y visualizar un \textit{dataset} completo,
escribir el manual de despliegue y los casos de uso.\\

La lista de tareas está disponible en
\href{https://github.com/IvanBeke/TFG-Visor-de-espectros/milestone/4?closed=1}{Sprint 3}\footnote{\url{https://github.com/IvanBeke/TFG-Visor-de-espectros/milestone/4?closed=1}}.
El manual de despliegue y la subida de \textit{datasets} no se pudieron
completar. El gráfico \textit{burndown} del \textit{sprint} se ve en la
figura~\ref{fig:burndown-sprint3}.\\

\imagen{burndown-sprint3}{Burndown del \textit{sprint} 3}

\subsection{Sprint 4}
En este \textit{sprint} se completan las tareas que no habían dado tiempo del
\textit{sprint} anterior y se planteó usar una base de datos en lugar de
almacenamiento para guardar los datos, por lo que se realizó una comparación
entre formas de almacenar los datos. También se añadió la opción de borrar un
\textit{dataset} ya almacenado.\\

Al igual que en el \textit{sprint} anterior se estructuró la aplicación como
conjunto, en este \textit{sprint} se estructura la parte de visualización.
También se arreglaron dos \textit{bugs} que se introdujeron en el
\textit{sprint} anterior en la aplicación desplegada.\\

La lista de tareas está disponible en
\href{https://github.com/IvanBeke/TFG-Visor-de-espectros/milestone/5?closed=1}{Sprint 4}\footnote{\url{https://github.com/IvanBeke/TFG-Visor-de-espectros/milestone/5?closed=1}}.
Todas las tareas se completaron a tiempo. El gráfico \textit{burndown} del
\textit{sprint} se ve en la figura~\ref{fig:burndown-sprint4}.\\

\imagen{burndown-sprint4}{Burndown del \textit{sprint} 4}

\subsection{Sprint 5}
De la comparación del \textit{sprint} anterior se decidió usar MongoDB para el
almacenamiento, por lo cual la mayoría de los esfuerzos se centraron en adaptar
la aplicación para usar MongoDB en todos sus aspectos: guardar, borrar y coger
los datos. También se solucionó un \textit{bug} en la parte de visualización.\\

La lista de tareas está disponible en
\href{https://github.com/IvanBeke/TFG-Visor-de-espectros/milestone/6?closed=1}{Sprint 5}\footnote{\url{https://github.com/IvanBeke/TFG-Visor-de-espectros/milestone/6?closed=1}}.
Todas las tareas se completaron a tiempo. El gráfico \textit{burndown} del
\textit{sprint} se ve en la figura~\ref{fig:burndown-sprint5}.\\

\imagen{burndown-sprint5}{Burndown del \textit{sprint} 5}

\subsection{Sprint 6}
Para este \textit{sprint} se planteó cambiar la forma en la que se guardan los
\textit{datasets}, de forma que cuando el sistema de aprendizaje automático esté
implementado sea más sencillo pasar los datos para el entrenamiento. También se
añadió soporte de comentarios en el \textit{dataset}. Debido a un cambio en como
la geóloga organizaba sus datos, se tuvo de adaptar la subida de
\textit{datasets} a este cambio. Por último se empezó a añadir controles en la
parte de visualización para el procesado de los espectros.\\

La lista de tareas está disponible en
\href{https://github.com/IvanBeke/TFG-Visor-de-espectros/milestone/7?closed=1}{Sprint 6}\footnote{\url{https://github.com/IvanBeke/TFG-Visor-de-espectros/milestone/7?closed=1}}.
La tarea de los controles no se completó a tiempo. El gráfico \textit{burndown}
del \textit{sprint} se ve en la figura~\ref{fig:burndown-sprint6}.\\

\imagen{burndown-sprint6}{Burndown del \textit{sprint} 6}

\subsection{Sprint 7}
Este \textit{sprint} se centró en completar la tarea del \textit{sprint}
anterior, cambiar la parte de visualización para que se muestre en un tabla los
ejemplos subidos y sus metadatos, documentar el código y avanzar en la memoria y
anexos.\\

Durante el \textit{sprint}, se vio que la tarea sobre procesamiento de datos era
demasiado extensa, dividiéndola en dos partes, la primera que sería añadir los
controles para el procesamiento en la interfaz, y una segunda para añadir el
código de procesamiento de datos, para realizar en el siguiente
\textit{sprint}.\\

La lista de tareas está disponible en
\href{https://github.com/IvanBeke/TFG-Visor-de-espectros/milestone/8?closed=1}{Sprint 7}\footnote{\url{https://github.com/IvanBeke/TFG-Visor-de-espectros/milestone/8?closed=1}}.
No dio tiempo a escribir la introducción de la memoria. El gráfico
\textit{burndown} del \textit{sprint} se ve en la
figura~\ref{fig:burndown-sprint7}.\\

\imagen{burndown-sprint7}{Burndown del \textit{sprint} 7}

\subsection{Sprint 8}
Este \textit{sprint} se centró en completar la tarea del \textit{sprint}
anterior, arreglar un \textit{bug} del despliegue, añadir al proyecto el código
de procesamiento de datos y enlazarlo a los controles, añadir instrucciones de
esta parte y ampliar la planificación temporal con links al repositorio y
capturas de los gráficos \textit{burndown}.\\

Se recuerda que el código de procesamiento es gran parte de los resultados del
proyecto previo en las colaboraciones.\\

La lista de tareas está disponible en
\href{https://github.com/IvanBeke/TFG-Visor-de-espectros/milestone/9?closed=1}{Sprint 8}\footnote{\url{https://github.com/IvanBeke/TFG-Visor-de-espectros/milestone/9?closed=1}}.
Todas las tareas se completaron a tiempo. El gráfico
\textit{burndown} del \textit{sprint} se ve en la
figura~\ref{fig:burndown-sprint8}.\\

\imagen{burndown-sprint8}{Burndown del \textit{sprint} 8}

\subsection{Sprint 9}
Este \textit{sprint} se centró en añadir la funcionalidad de los espectros
individuales: subida, visualización, procesado y borrado. También se trabajó en
los objetivos de la memoria.\\

La lista de tareas está disponible en
\href{https://github.com/IvanBeke/TFG-Visor-de-espectros/milestone/10?closed=1}{Sprint 9}\footnote{\url{https://github.com/IvanBeke/TFG-Visor-de-espectros/milestone/10?closed=1}}.
Todas las tareas se completaron a tiempo. El gráfico
\textit{burndown} del \textit{sprint} se ve en la
figura~\ref{fig:burndown-sprint9}.\\

\imagen{burndown-sprint9}{Burndown del \textit{sprint} 9}

\section{Estudio de viabilidad}

\subsection{Viabilidad económica}

\subsection{Viabilidad legal}
