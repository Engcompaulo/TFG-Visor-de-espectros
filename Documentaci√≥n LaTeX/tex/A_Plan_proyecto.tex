\apendice{Plan de Proyecto Software}

\section{Introducción}
Para que un proyecto se desarrolle con normalidad y con el menor número de
imprevistos posibles es esencial que cuente con una fase de planificación. Aquí
se estima el tiempo, trabajo y dinero necesario para realización del proyecto.

Para ello, se debe analizar en detalle cada parte del proyecto. De cara al 
futuro, el análisis del proyecto puede servir para predecir como de bien puede 
desarrollarse una continuación del mismo.

La planificación del proyecto consta de dos partes:
\begin{itemize}
	\tightlist
	\item \textbf{Planificación temporal}: en esta parte se analiza y planifica el
	tiempo que se va a dedicar a cada parte del proyecto, fecha de inicio y final
	aproximado, teniendo en cuenta el trabajo necesario para cada parte.
	\item \textbf{Estudio de viabilidad}: en esta parte se analiza como de viable
	es la realización del proyecto, se divide a su vez en dos apartados:
	
	\begin{itemize}
		\tightlist
		\item \textbf{Económica}: en esta parte se estiman los costes y los beneficios
		que puede suponer el proyecto.
		\item \textbf{Legal}: en esta parte se analizan los conceptos legales del
		proyecto, como podrían ser las licencias del proyecto o la política de
		protección de datos.
	\end{itemize}
\end{itemize}

\section{Planificación temporal}
La planificación temporal se organiza mediante \textit{sprints}. Cada
\textit{sprint} dura una o dos semanas. Al terminar cada \textit{sprint} se
realiza una reunión con los tutores para darlo por terminado. En los primeros 
\textit{sprints} no aparecen los gráficos debido a que, al ser principalmente 
de investigación, no se estimaban las tareas, y sin la estimación no se puede 
generar el grafico.

\subsection{Sprint 0}
En este \textit{sprint} se marcó el inicio del proyecto. La lista de tareas está
disponible en
\hrefFootnote{https://github.com/IvanBeke/TFG-Visor-de-espectros/milestone/1?closed=1}{Sprint
0}. En reuniones previas se habló con los tutores en que iba a consistir en
proyecto, pero no estaba claro con que tecnologías desarrollarlo. Se decidió
hacer una evaluación de las tecnologías posibles y crear unos prototipos
básicos. Todas las tareas se completaron a tiempo.\\

\subsection{Sprint 1}
En este \textit{sprint} se habló sobre el despliegue de la aplicación. Los
tutores comentaron que en proyectos webs anteriores el despliegue se solía dejar
para las etapas finales del proyecto, haciendo que todos los problemas asociados
surjan en esas etapas finales, retrasando el despliegue y, a veces, impidiendo
desplegar la aplicación.\\

Se conocía la plataforma Heroku así que fue la primera opción que se probó., 
adicionalmente se buscaron otras alternativas. También se usó el prototipo 
escogido para crear el proyecto definitivo y se trabajó en la memoria. La 
última parte fue estudiar las guías de estilo de Python, aplicarlas en los 
prototipos y documentar su código.\\

La lista de tareas está disponible en
\hrefFootnote{https://github.com/IvanBeke/TFG-Visor-de-espectros/milestone/2?closed=1}{Sprint
1}. Todas las tareas se completaron a tiempo.\\

\subsection{Sprint 2}
Dado que en la aplicación se necesita que los usuarios suban contenido, se
necesita control de usuarios, en este \textit{sprint} se investigaron formas de
ofrecerlo. También debido a la necesidad de disponer almacenamiento persistente,
se tuvo que mirar otras formas de despliegue e investigar como Heroku lo 
ofrece, dado que por defecto no lo hace. Al final se decidió cambiar a 
Digital Ocean con Nanobox.\\

La lista de tareas está disponible en
\hrefFootnote{https://github.com/IvanBeke/TFG-Visor-de-espectros/milestone/3?closed=1}{Sprint
2}. Todas las tareas se completaron a tiempo.\\

\subsection{Sprint 3}
En este \textit{sprint} se podría decir que comienza el desarrollo del proyecto,
basado en el prototipo. Como tal se mejoró el aspecto visual de la aplicación,
se cambio la estructura para tener partes diferenciadas y mantenibles, añadir
control de usuarios y mejorar la subida de ficheros.\\

También se planteó añadir, subir y visualizar un \textit{dataset} completo,
escribir el manual de despliegue y los casos de uso.\\

La lista de tareas está disponible en
\hrefFootnote{https://github.com/IvanBeke/TFG-Visor-de-espectros/milestone/4?closed=1}{Sprint
3}. El manual de despliegue y la subida de \textit{datasets} no se pudieron
completar. El gráfico \textit{burndown} del \textit{sprint} se ve en la
figura~\ref{fig:burndown-sprint3}.\\

\imagen{burndown-sprint3}{Burndown del \textit{sprint} 3}

\subsection{Sprint 4}
En este \textit{sprint} se completan las tareas que no habían dado tiempo del
\textit{sprint} anterior y se planteó usar una base de datos en lugar de
almacenamiento para guardar los datos, por lo que se realizó una comparación
entre formas de almacenar los datos. También se añadió la opción de borrar un
\textit{dataset} ya almacenado.\\

Similar a lo ocurrido en el \textit{sprint} anterior, se cambió la estructura 
de la aplicación, en este \textit{sprint} se estructuró la parte de 
visualización. También se arreglaron dos errores que se introdujeron en el
\textit{sprint} anterior en la aplicación desplegada.\\

La lista de tareas está disponible en
\hrefFootnote{https://github.com/IvanBeke/TFG-Visor-de-espectros/milestone/5?closed=1}{Sprint
4}. Todas las tareas se completaron a tiempo. El gráfico \textit{burndown} del 
\textit{sprint} se ve en la figura~\ref{fig:burndown-sprint4}.\\

\imagen{burndown-sprint4}{Burndown del \textit{sprint} 4}

\subsection{Sprint 5}
De la comparación del \textit{sprint} anterior se decidió usar MongoDB para el
almacenamiento, por lo cual la mayoría de los esfuerzos se centraron en adaptar
la aplicación para usar MongoDB en todos sus aspectos: guardar, borrar y coger
los datos. También se solucionó un error en la parte de visualización.\\

La lista de tareas está disponible en
\hrefFootnote{https://github.com/IvanBeke/TFG-Visor-de-espectros/milestone/6?closed=1}{Sprint
5}. Todas las tareas se completaron a tiempo. El gráfico \textit{burndown} del 
\textit{sprint} se ve en la figura~\ref{fig:burndown-sprint5}.\\

\imagen{burndown-sprint5}{Burndown del \textit{sprint} 5}

\subsection{Sprint 6}
Para este \textit{sprint} se planteó cambiar la forma en la que se guardan los
\textit{datasets}, de forma que cuando el sistema de aprendizaje automático esté
implementado, sea más sencillo pasar los datos para el entrenamiento. También se
añadió soporte de comentarios en el \textit{dataset}. Debido a un cambio en como
la geóloga organizaba sus datos, se tuvo que adaptar la subida de
\textit{datasets} a este cambio. Por último, se empezó a añadir controles en la
parte de visualización para el procesado de los espectros.\\

La lista de tareas está disponible en
\hrefFootnote{https://github.com/IvanBeke/TFG-Visor-de-espectros/milestone/7?closed=1}{Sprint
6}. La tarea de los controles no se completó a tiempo. El gráfico 
\textit{burndown} del \textit{sprint} se ve en la 
figura~\ref{fig:burndown-sprint6}.\\

\imagen{burndown-sprint6}{Burndown del \textit{sprint} 6}

\subsection{Sprint 7}
Este \textit{sprint} se centró en completar la tarea del \textit{sprint}
anterior, cambiar la parte de visualización para que se muestre en un tabla los
ejemplos subidos y sus metadatos, documentar el código y avanzar en la memoria y
anexos.\\

Durante el \textit{sprint}, se vio que la tarea sobre procesamiento de datos era
demasiado extensa, dividiéndola en dos partes, la primera que sería para añadir 
los controles para el procesamiento en la interfaz, y una segunda para añadir el
código de procesamiento de datos, para realizar en el siguiente
\textit{sprint}.\\

La lista de tareas está disponible en
\hrefFootnote{https://github.com/IvanBeke/TFG-Visor-de-espectros/milestone/8?closed=1}{Sprint
7}. No dio tiempo a escribir la introducción de la memoria. El gráfico
\textit{burndown} del \textit{sprint} se ve en la
figura~\ref{fig:burndown-sprint7}.\\

\imagen{burndown-sprint7}{Burndown del \textit{sprint} 7}

\subsection{Sprint 8}
Este \textit{sprint} se centró en completar la tarea del \textit{sprint}
anterior, arreglar un error del despliegue, añadir al proyecto el código
de procesamiento de datos y enlazarlo a los controles, añadir instrucciones de
esta parte y ampliar la planificación temporal con links al repositorio y
capturas de los gráficos \textit{burndown}.\\

Se recuerda que el código de procesamiento es gran parte de los resultados del
proyecto previo en las colaboraciones.\\

La lista de tareas está disponible en
\hrefFootnote{https://github.com/IvanBeke/TFG-Visor-de-espectros/milestone/9?closed=1}{Sprint
8}. Todas las tareas se completaron a tiempo. El gráfico
\textit{burndown} del \textit{sprint} se ve en la
figura~\ref{fig:burndown-sprint8}.\\

\imagen{burndown-sprint8}{Burndown del \textit{sprint} 8}

\subsection{Sprint 9}
Este \textit{sprint} se centró en añadir la funcionalidad de los espectros
individuales: subida, visualización, procesado y borrado. También se trabajó en
los objetivos de la memoria.\\

La lista de tareas está disponible en
\hrefFootnote{https://github.com/IvanBeke/TFG-Visor-de-espectros/milestone/10?closed=1}{Sprint
9}. Todas las tareas se completaron a tiempo. El gráfico
\textit{burndown} del \textit{sprint} se ve en la
figura~\ref{fig:burndown-sprint9}.\\

\imagen{burndown-sprint9}{Burndown del \textit{sprint} 9}

\subsection{Sprint 10}
Este \textit{sprint} se centró en añadir la funcionalidad de minería de datos
respecto a creación de clasificadores y varias mejoras visuales de la
interfaz.\\

Este \textit{sprint} duró dos semanas por desarrollarse a la vez que la época de
exámenes. Las tareas relacionadas con la mejora de la web no hubo problemas en
completarlas, pero las relacionadas a la creación de clasificadores llevaron más
tiempo del esperado y no se pudieron completar a tiempo.\\

La lista de tareas está disponible en
\hrefFootnote{https://github.com/IvanBeke/TFG-Visor-de-espectros/milestone/11?closed=1}{Sprint
10}. El gráfico \textit{burndown} del \textit{sprint} se ve en la
figura~\ref{fig:burndown-sprint10}.\\

\imagen{burndown-sprint10}{Burndown del \textit{sprint} 10}

\subsection{Sprint 11}
En este \textit{sprint} se completaron las tareas pendientes del \textit{sprint}
anterior, además se realizó también la tarea de codificar la predicción de 
nuevos espectros. También se avanzó bastante en la memoria, aunque no se 
llegaron a terminar completamente todas las secciones previstas.\\

Debido a la cercanía de la entrega este \textit{sprint} tuvo más carga de
trabajo que los anteriores. Se puede considerar el último \textit{sprint} de
desarrollo porque se terminaron de implementar los objetivos del proyecto.\\

La lista de tareas está disponible en
\hrefFootnote{https://github.com/IvanBeke/TFG-Visor-de-espectros/milestone/12?closed=1}{Sprint
11}. El gráfico \textit{burndown} del \textit{sprint} se ve en la
figura~\ref{fig:burndown-sprint11}. Debido a que gran parte de las tareas eran
de documentación, se avanzaba en ellas en paralelo y se cerraron al final del
\textit{sprint}, de ahí la forma del gráfico.\\

\imagen{burndown-sprint11}{Burndown del \textit{sprint} 11}

\subsection{Sprint 12}

Último \textit{sprint} del proyecto. Se realizaron mejorar menores en la 
aplicación, se implementaron unos tests unitarios y se terminó de escribir la 
memoria y anexos. Se completaron todas las tareas.

La lista de tareas está disponible en
\hrefFootnote{https://github.com/IvanBeke/TFG-Visor-de-espectros/milestone/13?closed=1}{Sprint
 12}. El gráfico \textit{burndown} del \textit{sprint} se ve en la 
figura~\ref{fig:burndown-sprint12}.

\imagen{burndown-sprint12}{Burndown del \textit{sprint} 12}

\section{Estudio de viabilidad}

\subsection{Viabilidad económica}

\subsubsection{Costes de personal}

El proyecto se ha llevado a cabo por un desarrollador contratado a tiempo 
parcial durante 4 meses. Se considera un salario neto de \EUR{1000} mensuales 
(ver tabla~\ref{tab:personal}).

La cotización a la seguridad social se ha calculado como horas comunes, según 
el régimen general de 2018 (28,30\%)~\cite{seguridad-social}.

\begin{table}[!h]
	\centering
	\begin{tabular}{@{}l|l@{}}
		\toprule
		\textbf{Concepto} & \textbf{Coste} \\
		\midrule
		Salario neto & \EUR{1000}  \\
		Retención IRPF (19 \%) & \EUR{360,53} \\
		Seguridad social (28,30 \%) & \EUR{537,00} \\
		\midrule
		Salario bruto (mensual) & \EUR{1897,53} \\
		\midrule
		\textbf{Total 4 meses} & \EUR{7590,12} \\
		\bottomrule
	\end{tabular}
	\caption{Costes de personal}
	\label{tab:personal}
\end{table}

Además se sumará el sueldo de los dos tutores asignados al proyecto durante 4 
meses, que corresponde a 0,5 créditos \cite{misc:retribuciontutores-laboral, 
misc:retribuciontutores-funcionarios}.

\begin{itemize}
	\item Ayudante Doctor. Sueldo mensual: \EUR{1.815,61}. Imparte 24 créditos 
	anuales.
	$$ \dfrac{\textup{\EUR{1.815,61}} * 12 \;meses * 0.5 \;créditos}{24 
	\;créditos} = \textup{\EUR{453,90}} $$
	\item Profesor Titular de Universidad. Sueldo mensual: \EUR{3.527,89}. 
	Imparte 24 créditos anuales.
	$$ \dfrac{\textup{\EUR{3.527,89}} * 12 \;meses * 0.5 \;créditos}{24 
	\;créditos} = \textup{\EUR{881,97}} $$
\end{itemize}

El coste de los tutores corresponde a \EUR{1335,87}


\subsubsection{Costes de hardware}
En esta sección se enumeran los costes del hardware usado durante el desarrollo.

Para el desarrollo se ha usado un ordenador portátil valorado en \EUR{800}, con 
amortización en 4 años (ver tabla~\ref{tab:hardware}).

$$\dfrac{\textup{\EUR{800}}}{4 \;años * 12 \;meses} = 
\textup{\EUR{16,67}} $$

\begin{table}[!h]
	\centering
	\begin{tabular}{@{}l|l|l@{}}
		\toprule
		\textbf{Concepto} & \textbf{Coste} & \textbf{Amortización} \\
		\midrule
		Ordenador portátil & \EUR{800} & \EUR{16,67} \\
		\midrule
		\textbf{Total 4 meses} & \EUR{66,67} \\
		\bottomrule
	\end{tabular}
	\caption{Costes de hardware}
	\label{tab:hardware}
\end{table}

\subsubsection{Costes de software}
El proyecto se ha desarrollado usando el sistema Ubuntu, por lo que en este 
aspecto no habría costes. La licencia de PyCharm Professional supone un gasto 
de \EUR{8,90} mensuales~\cite{pycharm-price}. La licencia de GitKraken Pro 
supone un pago único anual de 49\$, que aproximadamente corresponde a 
\EUR{42}~\cite{gitkraken-price}. Los costes se resumen en la 
tabla~\ref{tab:software}.

\begin{table}[!h]
	\centering
	\begin{tabular}{@{}l|l@{}}
		\toprule
		\textbf{Concepto} & \textbf{Coste} \\
		\midrule
		PyCharm Professional (mensual) & \EUR{8,90} \\
		GitKraken Pro  &  \EUR{42} \\
		\midrule
		\textbf{Total 4 meses} & \EUR{77.60} \\
		\bottomrule
	\end{tabular}
	\caption{Costes de software}
	\label{tab:software}
\end{table}

\subsubsection{Costes del servidor}
Como servidor se ha elegido un \textit{Droplet} estándar con 2GB de memoria, 1 
CPU virtual, 50GB de disco duro y 2TB de transferencia cuyo coste es de 10\$ 
mensuales, aproximadamente \EUR{8,60} mensuales~\cite{digital-price}. Los 
costes se resumen en la tabla~\ref{tab:servidor}.

\begin{table}[!h]
	\centering
	\begin{tabular}{@{}l|l@{}}
		\toprule
		\textbf{Concepto} & \textbf{Coste} \\
		\midrule
		Digital Ocean (mensual) & \EUR{8,60} \\
		\midrule
		\textbf{Total 4 meses} & \EUR{34,40} \\
		\bottomrule
	\end{tabular}
	\caption{Costes del servidor}
	\label{tab:servidor}
\end{table}

\subsubsection{Costes totales}

En la tabla~\ref{tab:total} se agrupan todos los costes calculados del 
proyecto, dando el total de \EUR{7768,99}.

\begin{table}[!h]
	\centering
	\begin{tabular}{@{}l|l@{}}
		\toprule
		\textbf{Concepto} & \textbf{Coste} \\
		\midrule
		Personal & \EUR{7590,12} \\
		Tutores  & \EUR{1335,87} \\
		Hardware & \EUR{66,67} \\
		Software & \EUR{77,60} \\
		Servidor & \EUR{34,60} \\
		\midrule
		\textbf{Total} & \EUR{9104,86} \\
		\bottomrule
	\end{tabular}
	\caption{Costes totales del proyecto}
	\label{tab:total}
\end{table}

\subsubsection{Beneficios}

Para obtener beneficios del proyecto se podría plantear añadir a la aplicación 
las siguientes alternativas:
\begin{itemize}
	\item Limitaciones de almacenamiento: la cantidad de ficheros que puede 
	subir un usuario estaría limitada por un plan de pago.
	\item \textit{Freemium}~\cite{wiki:freemium}: este modelo funciona 
	ofreciendo unas funcionalidades básicas a todos los usuarios, pero 
	bloqueando algunas a usuarios que no hayan pagado, por ejemplo, la creación 
	de clasificadores.
	\item Publicidad: podría incluirse en la aplicación publicidad relacionado 
	con el tema.
\end{itemize}

Para obtener el máximo beneficio estas opciones podrían combinarse.

\subsection{Viabilidad legal}

A lo hora de añadir una licencia al proyecto hay que tener en cuenta a que 
licencias están sometidas las dependencias. Con la ayuda de la herramienta 
\hrefFootnote{https://requires.io/}{Requires.io}, se han listado las 
dependencias, versión usada y licencia (ver tabla~\ref{tab:dependencias}).

\begin{table}[]
	\centering
	\begin{tabular}{lll}
		\hline
		Dependencia            & Versión & Licencia           \\
		\hline
		dash                   & 0.21.1  & MIT                \\
		dash-core-components   & 0.22.1  & MIT                \\
		dash-html-components   & 0.10.1  & MIT                \\
		dash-table-experiments & 0.6.0   & MIT                \\
		dash\_renderer         & 0.11.3  & MIT                \\
		Flask                  & 1.0.2   & BSD                \\
		Flask-Bootstrap        & 3.3.7.1 & BSD                \\
		Flask-Dance            & 0.14.0  & MIT                \\
		Flask-PyMongo          & 0.5.1   & BSD                \\
		Flask-WTF              & 0.14.2  & BSD                \\
		gunicorn               & 19.8.1  & MIT                \\
		ipython                & 6.3.1   & BSD                \\
		numpy                  & 1.14.3  & BSD                \\
		numpydoc               & 0.8.0   & BSD                \\
		pandas                 & 0.23.0  & BSD                \\
		plotly                 & 2.5.1   & MIT                \\
		pymongo                & 3.6.1   & Apache License 2.0 \\
		pyOpenSSL              & 17.5.0  & Apache License 2.0 \\
		PyWavelets             & 0.5.2   & MIT                \\
		scikit-learn           & 0.19.1  & BSD 3-Clause       \\
		scipy                  & 1.1.0   & BSD                \\
		Werkzeug               & 0.14.1  & BSD                \\
		WTForms                & 2.2.1   & BSD                \\
		xlrd                   & 1.1.0   & BSD                \\
		\hline
	\end{tabular}
	\caption{Dependencias del proyecto}
	\label{tab:dependencias}
\end{table}

De las dependencias usadas, todas son licencias bastante permisivas que 
permiten el uso con libertad, por lo que no tenemos que preocuparnos de 
incompatibilidades con la licencia que se escoja.

Con ayuda de las recomendaciones GNU~\cite{gnu-choose}, se ha escodigo la 
licencia GPL-3.0~\cite{gpl3}. Esta licencia permite la modificación, uso y 
distribución del software, siempre que esto se haga bajo la misma licencia, se 
indiquen los cambios y se mencione al autor original.

