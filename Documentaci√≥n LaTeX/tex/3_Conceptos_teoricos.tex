\capitulo{3}{Conceptos teóricos}

\section{Espectroscopia Raman}\label{def:raman}
La espectroscopia Raman hace uso del fenómeno conocido como dispersión
inelástica de fotones para obtener gráficas que definen la estructura y
composición de un material o elemento.

Este fenómeno se refiere a como los fotones rebotan, o mejor dicho, son
absorbidos y vueltos a emitir. Los fotones se pueden
dispersar de forma elástica (Rayleigh) o inelástica (Raman).

En la primera forma los fotones absorbidos son emitidos de vuelta igual que
fueron absorbidos, la gran mayoría de ellos, pero una pequeña cantidad se emiten
cambiados, con una pequeña disminución o aumento de sus energía (ver
figura~\ref{fig:dispersion-raman}).

Este cambio de energía varía según el material o elemento contra el que impacten
los fotones, revelando ahí la estructura o composición y abriendo un amplio
abanico de aplicaciones para esta
técnica~\cite{what-is-raman,wiki:raman-scatter}. Con la dispersión Raman
capturada se obtienen los espectros Raman.

\imagen{dispersion-raman}{Representración de la dispersión de
	fotones~\cite{what-is-raman}}

Con lo explicado anteriormente se puede ver que esta técnica tiene varias
ventajas, entre las que destacan~\cite{what-is-raman}:
\begin{itemize}
	\tightlist
	\item Análisis sin contacto y no destructivo.
	\item No se suele necesitar preparar la muestra
	\item Sirve para materia orgánica e inorgánica.
	\item Se puede usar para elementos en cualquier estado de la materia
	\item Para conseguir un espectro la exposición de la muestra al láser está
	entre 10ms y 1s.
\end{itemize}

\section{Visualización de datos}

Técnicas usadas con el propósito de presentar datos o información mediante
gráficos, como puntos, líneas o barras. Está considerado uno de los pasos dentro
del análisis de datos o ciencia de datos\cite{wiki:dataviz}.

Su objetivo principal es el de comunicas información de forma clara y eficiente,
sin significar esto que para que un gráfico sea funcional tenga que parecer
aburrido ni que tenga que ser extremadamente sofisticado para resultar
agradable\cite{friedman2008}.

\section{Minería de datos}

La minería de datos se define como la aplicación de técnicas de inteligencia
artificial sobre grandes cantidades de datos, con el objetivo de descubrir
tendencias, patrones o relaciones ocultas.

Estos descubrimientos suelen ser usados para describir, resumir o clasificar en
grupos un conjunto de datos, además de para predecir en que grupo del conjunto
encajarían nuevos ejemplos de los datos.

Las fases del proceso de minería de datos se pueden dividir en las siguientes:
\begin{enumerate}
	\item \textbf{Selección}: a partir del conjunto de datos original seleccionar
	los ejemplos con los que se va a trabajar.
	\item \textbf{Preprocesamiento}: aplicar operaciones sobre los datos para
	eliminar ruido o medidas erróneas.
	\item \textbf{Transformación}: transformar los datos preprocesados a un formato
	sobre el que poder aplicar las técnicas de minería de datos.
	\item \textbf{Minería de datos}: aplicar algoritmos sobre los datos capaces de
	extraer patrones ocultos en ellos.
	\item \textbf{Evaluación}: comprobar como de bien funcionan los patrones
	descubiertos en datos nuevos.
\end{enumerate}

Este proyecto se centra en la parte de preprocesamiento, minería de datos y
evaluación.

\subsection{Preprocesamiento}

En esta sección se definen las operaciones de preprocesamiento ofrecidas en la
aplicación. Se van a explicar tomando como ejemplo uno de los espectros usados
por la aplicación.

\subsubsection{Recorte (Crop)}

La operación de recorte devuelve los valores del espectro contenidos entre un
límite inferior y un límite superior (ver figura~\ref{fig:recorte}).

\imagen{recorte}{Recorte entre valores 200 y 1500}

\subsubsection{Corrección de línea base (Baseline correction)}

La línea base de un gráfico se puede definir como una línea imaginaria sobre la
que se apoyan los datos. Esta línea da lugar a mediciones incorrectas por lo que
es conveniente eliminarla de los gráficos. Como se ve en la
figura~\ref{fig:baseline}, antes de aplicar la operación, los datos van
ascendiendo a medida que aumenta el eje X, después de la operación los datos
están correctamente nivelados.

\imagen{baseline}{Corrección de la línea base}

\subsubsection{Normalización (Normalization)}

La normalización consiste en el proceso de transformar los valores de los
espectros de tal forma que todos los espectros se midan por la misma escala para
poder ser comparados. Como se ve en la figura~\ref{fig:norm} la escala ha
cambiado al rango (0, 1).

\imagen{norm}{Normalización de los datos}

\subsubsection{Compresión (Squash)}

Esta operación consiste en la construcción de datos similares a los originales
pero de menor tamaño, sin embargo estos datos nuevos deben producir un resultado
casi igual al original al ser analizados. Como se ve en la
figura~\ref{fig:squash}, el gráfico representado parece el mismo pero la escala
es menor, indicando la transformación de los datos.

\imagen{squash}{Compresión de los datos}

\subsubsection{Suavizado (Smooth)}

La operación de suavizado está enfocada a eliminar el ruido del espectro. Este
ruido provoca picos en el gráfico que impiden analizarlo bien ya que 
pueden modificar la altura de picos que interesan realmente. Esta operación
necesita de un parámetro que indica cuanto se van a reducir los picos (ver
figura~\ref{fig:smooth}).

\imagen{smooth}{Suavizado con ventana de tamaño 25}

\section{Base de datos NoSQL}

Las bases de datos no relacionales, o NoSQL (tradicionalmente ``non SQL'',
actualmente ``Not Only SQL'')\cite{wiki:nosql} son sistemas de almacenamiento de
datos que no requieren de una estructura definida y/o fija para su
funcionamiento.

Se diferencian principalmente de de los sistemas de bases de datos relacionales
en que no necesitan tener definido un esquema al que se ajusten los datos, si no
que estos pueden tener la estructura que necesiten y no necesariamente la misma
a otros datos almacenados.

Los principales tipos en los que se dividen este tipo de sistemas son:
\begin{itemize}
	\item Bases de datos documentales
	\item Bases de datos clave/valor
	\item Bases de datos en grafo
	\item Bases de datos orientadas a objetos
\end{itemize}

%\tablaSmall{Herramientas y tecnologías utilizadas en cada parte del proyecto}{l
%	c c c c}{herramientasportipodeuso}
%{ \multicolumn{1}{l}{Herramientas} & App AngularJS & API REST & BD & Memoria
%	\\}{ 
%	HTML5 & X & & &\\
%	CSS3 & X & & &\\
%	BOOTSTRAP & X & & &\\
%	JavaScript & X & & &\\
%	AngularJS & X & & &\\
%	Bower & X & & &\\
%	PHP & & X & &\\
%	Karma + Jasmine & X & & &\\
%	Slim framework & & X & &\\
%	Idiorm & & X & &\\
%	Composer & & X & &\\
%	JSON & X & X & &\\
%	PhpStorm & X & X & &\\
%	MySQL & & & X &\\
%	PhpMyAdmin & & & X &\\
%	Git + BitBucket & X & X & X & X\\
%	Mik\TeX{} & & & & X\\
%	\TeX{}Maker & & & & X\\
%	Astah & & & & X\\
%	Balsamiq Mockups & X & & &\\
%	VersionOne & X & X & X & X\\
%} 

