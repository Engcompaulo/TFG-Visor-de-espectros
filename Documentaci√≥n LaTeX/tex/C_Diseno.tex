\apendice{Especificación de diseño}

\section{Introducción}

En este apéndice se expone el diseño que ha dado lugar a la aplicación. Se 
incluye el diseño de datos, diseño procedimental y diseño arquitectónico.

\section{Diseño de datos}

En esta sección se explican las entidades usadas por la aplicación:

\begin{itemize}
	\item Dataset: esta entidad representa los conjuntos de espectros subidos 
	por los usuarios.
	\item Spectrum: esta entidad representa un espectro subido por los usuarios.
	\item ClassifierSet: esta entidad representa los clasificadores entrenados 
	por los usuarios, se compone de tres clasificadores, uno por cada atributo 
	a predecir.
\end{itemize}

\imagen{clases}{Diagrama de clases}

\section{Diseño procedimental}

En esta sección se explican las interacciones más importantes de la aplicación. 
Las interacción principal de la aplicación es, dentro de la visualización de 
datasets, cuando se selecciona un espectro para visualizar, ya que entran en 
juego todas las partes de la aplicación (ver figura~\ref{fig:carga-espectro}).

\imagen{carga-espectro}{Diagrama de secuencia de visualización de espectro}

Cuando el usuario selecciona un espectro, se envía al servidor la fila 
seleccionada, el servidor pide a la base de datos el dataset que se está 
visualizando y lo devuelve, el servidor extrae del dataset el espectro 
seleccionado, lo convierte en una figura que se pueda visualizar y lo envía a 
la interfaz, que se lo muestra al usuario.

A continuación, para mostrar el espectro procesado, la interfaz envía al 
servidor los valores actuales de los controles, el servidor aplica las acciones 
de procesamiento al espectro, lo convierte en una figura para poder 
visualizarlo y lo envía a la interfaz, que lo muestra al usuario.

\section{Diseño arquitectónico}

Al haber desarrollado una aplicación web, la arquitectura del proyecto está 
condicionada por ello.

\subsection{Modelo Vista Controlador (MVC)}

Se ha seguido el patrón de MVC con el objetivo de separar la lógica de la 
aplicación (controlador), los datos (modelo) y la interfaz (vista). Este patrón 
permite facilitar la tarea del mantenimiento. Además, al ser un patrón 
conocido, se facilita la tarea de trabajar en la aplicación a futuros 
desarrolladores. 

\imagen{mvc}{Diagrama del patrón MVC~\cite{wiki:mvc}}

Los modelos se encuentran presentes en \code{SpectraViewer/models.py}, los 
controladores se encuentran en \code{SpectraViewer/main/routes.py} y las vistas 
se encuentran en \code{SpectraViewer/templates/}.
