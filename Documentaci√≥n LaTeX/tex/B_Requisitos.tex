\apendice{Especificación de Requisitos}

\section{Introducción}

En este apéndice se describen los objetivos generales de la aplicación y se 
detallan sus requisitos, tanto funcionales como no funcionales.

\section{Objetivos generales}

\begin{itemize}
	\item Ofrecer control de usuarios.
	\item Permitir a los usuarios subir \textit{datasets} y espectros.
	\item Que los \textit{datasets} y espectros se puedan visualizar.
	\item Que sobre la visualización se pueda aplicar operaciones de 
	procesamiento.
	\item Poder entrenar modelos de aprendizaje automático con los
	\textit{datasets} subidos.
	\item Poder usar los modelos mencionados anteriormente para predecir nuevos
	ejemplos subidos.
	\item Que la aplicación final sea útil para la investigadora.
\end{itemize}

\section{Catalogo de requisitos}

\subsection{Requisitos funcionales}

\begin{itemize}
	\item \textbf{RF-1 Control de usuarios}: la aplicación debe permitir 
	controlar usuarios.
	\begin{itemize}
		\item \textbf{RF-1.1 Integración con Google}: la aplicación debe poder 
		hacer uso de cuentas de Google para el control de usuarios.
		\item \textbf{RF-1.2 Inicio de sesión}: el usuario debe poder iniciar 
		sesión con una	cuenta de Google.
		\item \textbf{RF-1.3 Cierre de sesión}: el usuario debe poder cerrar 
		sesión cuando haya terminado.
	\end{itemize}
	\item \textbf{RF-2 Datasets}: la aplicación debe poder almacenar y 
	gestionar conjuntos de espectros.
	\begin{itemize}
		\item \textbf{RF-2.1 Subida}: el usuario debe poder subir un 
		\textit{dataset}.
		\item \textbf{RF-2.2 Eliminado}: el usuario debe poder eliminar un 
		\textit{dataset} almacenado.
		\item \textbf{RF-2.3 Visualización}: el usuario debe poder visualizar 
		un \textit{dataset} almacenado.
	\end{itemize}
	\item \textbf{RF-3 Espectros}: la aplicación debe poder almacenar y 
	gestionar espectros.
	\begin{itemize}
		\item \textbf{RF-3.1 Subida}: el usuario debe poder subir un espectro.
		\item \textbf{RF-3.2 Eliminado}: el usuario debe poder eliminar un 
		espectro.
		\item \textbf{RF-3.3 Visualización}: el usuario debe poder visualizar 
		un espectro.
	\end{itemize}
	\item \textbf{RF-4 Procesamiento}: el usuario debe poder aplicar 
	operaciones de preprocesamiento.
	\begin{itemize}
		\item \textbf{RF-4.1: Procesamiento de \textit{dataset}}: el usuario 
		debe poder aplicar operaciones de preprocesamiento sobre un 
		\textit{dataset} visualizado.
		\item \textbf{RF-4.2: Procesamiento de espectro}: el usuario debe poder 
		aplicar operaciones de preprocesamiento sobre un espectro visualizado.
	\end{itemize}
	\item \textbf{RF-5 Minería de datos}: el usuario debe poder usar técnicas 
	de minería de datos.
	\begin{itemize}
		\item \textbf{RF-5.1 Creación}: el usuario debe poder crear modelos 
		personalizados.
		\item \textbf{RF-5.2 Evaluación}: la aplicación debe ofrecer métricas 
		del modelo entrenado.
		\item \textbf{RF-5.3 Predicción}: el usuario debe poder usar los 
		modelos que ha creado para predecir nuevos espectros.
	\end{itemize}
\end{itemize}

\subsection{Requisitos no funcionales}

	\begin{itemize}
	\item \textbf{RNF-1 Usabilidad}: la aplicación debe ser intuitiva y fácil 
	de usar.
	\item \textbf{RNF-2 Escalabilidad}: el rendimiento debe poder aumentar al 
	aumentar los recursos.
	\item \textbf{RNF-3 Manteninibilidad}: debe ser sencillo añadir 
	funcionalidad nueva a la aplicación.
	\item \textbf{RNF-4 Compatibilidad}: la aplicación debe poder funcionar en 
	los principales navegadores.
	\item \textbf{RNF-5 Responsividad}: la aplicación debe adaptarse al tamaño 
	de la pantalla.
	\item \textbf{RNF-6 Facilidad de despliegue}: la aplicación debe poder 
	desplegarse en un servidor de forma sencilla.
	\end{itemize}

\section{Especificación de requisitos}

En esta sección se presentan el diagrama de casos de uso. A continuación, se 
desarrollan los casos de uso mostrados en el diagrama.

\imagen{casos-uso}{Diagrama de casos de uso}

\casoDeUso{1}{Iniciar sesión}
{RF-1, RF-1.1, RF-1.2}
{El usuario inicia sesión en la aplicación}
{Navegador web abierto y página de la aplicación cargada}
{
	\item Pulsar botón ``Iniciar sesión con Google''.
	\item Introducir o seleccionar cuenta de Google.
}
{Redirección a la aplicación con sesión iniciada}
{	\item Cuenta no existente.
	\item Combinación de nombre y contraseña incorrecta.
}
{Alta}

\casoDeUso{2}{Cerrar sesión}
{RF-1, RF-1.3}
{El usuario cierra sesión en la aplicación}
{Sesión iniciada en la aplicación}
{
	\item Pulsar en el botón cuyo texto es el correo.
	\item Pulsar ``Cerrar sesión''.
}
{Redirección a la aplicación con sesión cerrada}
{	\item La sesión no estaba iniciada.
}
{Alta}

\casoDeUso{3}{Subir dataset}
{RF-2, RF-2.1}
{El usuario sube un dataset a la aplicación para su guardado}
{Sesión iniciada en la aplicación}
{
	\item Pulsar en el botón ``Subir dataset''.
	\item Descargar y rellenar la plantilla.
	\item Crear un fichero \code{.zip} con los datos y la plantilla.
	\item Rellenar el formulario de subida.
	\item Seleccionar el fichero creado.
	\item Presionar el botón ``Subir''.
}
{Redirección a la página de los archivos guardados}
{	\item El formato del dataset no es correcto.
	\item Existe un dataset con el mismo nombre.
}
{Alta}

\casoDeUso{4}{Subir espectro}
{RF-3, RF-3.1}
{El usuario sube un espectro a la aplicación para su guardado}
{Sesión iniciada en la aplicación}
{
	\item Pulsar en el botón ``Subir espectro''.
	\item Rellenar el formulario de subida.
	\item Seleccionar el espectro.
	\item Presionar botón ``Subir''.
}
{Redirección a la página de los archivos guardados}
{	\item El formato del espectro no es correcto.
	\item Existe un espectro con el mismo nombre.
}
{Alta}

\casoDeUso{5}{Visualizar dataset}
{RF-2, RF-2.3, RF-4, RF-4.1}
{El usuario visualiza un espectro en la aplicación}
{Sesión iniciada en la aplicación, dataset guardado en la aplicación y estar en 
la página de ``Mis ficheros''}
{
	\item Escoger un dataset que visualizar.
	\item Pulsar el botón ``Visualizar'' en el dataset escogido.
}
{Se muestra la página de visualización}
{	\item No se ha podido cargar el dataset.
}
{Alta}

\casoDeUso{6}{Visualizar espectro}
{RF-3, RF-3.3, RF-4, RF-4.2}
{El usuario visualiza un espectro en la aplicación}
{Sesión iniciada en la aplicación, espectro guardado en la aplicación y estar 
en la página de ``Mis ficheros''}
{
	\item Escoger un espectro que visualizar.
	\item Pulsar el botón ``Visualizar'' en el espectro escogido.
}
{Se muestra la página de visualización}
{	\item No se ha podido cargar el espectro.
}
{Alta}

\casoDeUso{7}{Guardar clasificador}
{RF-5, RF-5.1, RF-5.2}
{El usuario crear y entrena un modelo usando como datos un dataset guardado}
{Sesión iniciada en la aplicación, dataset guardado en la aplicación y estar en 
la página de ``Mis ficheros''}
{
	\item Escoger el dataset que se quiera usar como base.
	\item Pulsar el botón ``Crear modelo'' en el dataset escogido.
	\item Seleccionar en el desplegable el modelo a usar.
	\item (Opcional) Rellenar el formulario con los parámetros del modelo.
	\item Pulsar el botón ``Crear y evaluar modelo''.
	\item Para guardar el clasificador:
	\begin{enumerate}
		\item Completar el formulario.
		\item Pulsar el botón ``Guardar''.
	\end{enumerate}
	\item Para no guardar el clasificador:
	\begin{enumerate}
		\item Pulsar el botón ``Descartar''.
	\end{enumerate}
}
{Se redirige a los archivos guardados}
{	\item Los parámetros del modelo introducidos son erróneos.
	\item Ya existe un clasificador con el nombre introducido.
}
{Media}

\casoDeUso{8}{Predecir espectro}
{RF-5, RF-5.3}
{El usuario usa un clasificador creado para predecir un espectro.}
{Sesión iniciada en la aplicación, dataset guardado en la aplicación, algún 
clasificador creado y estar en la página de ``Mis ficheros''}
{
	\item Escoger el espectro que se quiera predecir.
	\item Pulsar el botón ``Predecir'' en el espectro escogido.
	\item Seleccionar en el desplegable el clasificador a usar.
	\item Pulsar el botón ``Predecir espectro''.
	\item Para guardar el clasificador:
}
{Se muestra la predicción}
{	\item No se puede cargar el clasificador.
}
{Media}

