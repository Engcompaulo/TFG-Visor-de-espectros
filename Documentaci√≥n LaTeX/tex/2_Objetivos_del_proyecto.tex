\capitulo{2}{Objetivos del proyecto}

Este apartado explica de forma precisa y concisa cuales son los objetivos que se persiguen con la realización del proyecto. Se puede distinguir entre los objetivos marcados por los requisitos del software a construir y los objetivos de carácter técnico que plantea a la hora de llevar a la práctica el proyecto.

En esta sección se listan los objetivos que persigue la realización del proyecto, tanto 

\subsection{Objetivos principales}
\begin{itemize}
	\tightlist
	\item Desarrollar una aplicación que proporcione las opciones de procesamiento más comunes.
	\item Que la aplicación final sea útil para la investigadora.
	\item La aplicación tiene que ser usable e intuitiva.
	\item Que la herramienta se consiga desplegar.
	\item Ofrecer control de usuarios para que cada uno pueda almacenar sus archivos.
	\item Ofrecer un sistema de aprendizaje automático para ayudar en la clasificación de futuras muestras de espectros.
\end{itemize}

\subsection{Objetivos técnicos}
\begin{itemize}
	\tightlist
	\item Que la aplicación sea fácil de mantener.
	\item Utilizar git como sistema de control de versiones junto con GitHub para el repositorio remoto.
	\item Utilizar una metodología ágil, Scrum, para el desarrollo.
	\item Utilizar un sistema kanban para la gestión de tareas.
	\item Utilizar un sistema de revisión automática de código para asegurar su calidad.
	\item Utilizar un sistema de integración continua.
\end{itemize}

\subsection{Objetivos personales}
\begin{itemize}
	\tightlist
	\item Ampliar los conocimientos sobre desarrollo web a partir de los obtenidos durante el grado.
	\item Ampliar y profundizar conocimientos sobre Python, especialmente en desarrollo web y aprendizaje automático.
	\item Aprender a usar técnicas de aprendizaje automático en un entorno de investigación real.
	\item Desarrollar el proyecto de la forma más profesional posible.
\end{itemize}