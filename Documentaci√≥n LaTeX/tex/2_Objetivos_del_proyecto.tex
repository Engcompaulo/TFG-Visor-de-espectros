\capitulo{2}{Objetivos del proyecto}

Este apartado explica de forma precisa y concisa cuales son los objetivos que se
persiguen con la realización del proyecto. Se puede distinguir entre los
objetivos marcados por los requisitos del software a construir y los objetivos
de carácter técnico que plantea a la hora de llevar a la práctica el proyecto,
además de objetivos planteados a nivel personal.

\subsection{Objetivos principales}
\begin{itemize}
	\tightlist
	\item Ofrecer control de usuarios.
	\item Permitir a los usuarios subir \textit{datasets} y espectros.
	\item Que los \textit{datasets} y espectros se puedan visualizar.
	\item Que sobre la visualización se pueda aplicar operaciones de procesamiento.
	\item Poder entrenar modelos de aprendizaje automático con los
	\textit{datasets} subidos.
	\item Poder usar los modelos mencionados anteriormente para predecir nuevos
	ejemplos subidos.
	\item Que la aplicación final sea útil para la investigadora.
\end{itemize}

\subsection{Objetivos técnicos}
\begin{itemize}
	\tightlist
	\item Que la aplicación esté desplegada durante todo el desarrollo.
	\item Que la aplicación sea fácil de mantener.
	\item Utilizar git como sistema de control de versiones junto con GitHub como
	repositorio remoto.
	\item Utilizar una metodología ágil, Scrum, para el desarrollo.
	\item Utilizar un sistema kanban para la gestión de tareas.
	\item Utilizar un sistema de revisión automática de código para asegurar su
	calidad.
\end{itemize}

\subsection{Objetivos personales}
\begin{itemize}
	\tightlist
	\item Ampliar los conocimientos sobre desarrollo web a partir de los obtenidos
	durante el grado.
	\item Ampliar y profundizar conocimientos sobre Python, especialmente en
	desarrollo web y aprendizaje automático.
	\item Aprender a usar técnicas de aprendizaje automático en un entorno de
	investigación real.
	\item Desarrollar el proyecto de la forma más profesional posible.
\end{itemize}
