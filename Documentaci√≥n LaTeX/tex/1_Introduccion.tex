\capitulo{1}{Introducción}

La espectroscopia Raman es una técnica es espectroscópica basada en la dispersión inelástica de fotones, o dispersión Raman, de la luz monocromática disparada comúnmente desde un láser\cite{raman-basics}, generalmente en el rango de la luz visible o cercano al infrarrojo o ultravioleta\cite{wiki:raman-en}.

La dispersión de fotones se refiere a como los fotones lanzados por el láser rebotan, mejor dicho, son absorbidos y vueltos a emitir. Los fotones se pueden dispersar de forma elástica (Rayleigh) o inelástica (Raman).

En la primera forma los fotones absorbidos son emitidos de vuelta igual que fueron absorbidos, la gran mayoría de ellos, pero una pequeña se emiten cambiados, con un pequeño cambio en la su energía, ya sea a más o menos (ver Figura~\ref{fig:dispersion-raman}).

Este cambio de energía varía según el material o elemento contra el que impacten los fotones, revelando ahí la estructura o composición y abriendo un amplio abanico de aplicaciones para esta técnica\cite{what-is-raman,wiki:raman-scatter}. Con la dispersión Raman capturada se obtienen los espectros Raman.

\imagen{dispersion-raman}{Representración de la dispersión de fotones\cite{what-is-raman}}

Con lo explicado anteriormente se puede ver que esta técnica tiene varias ventajas, entre las que destacan\cite{what-is-raman}:
\begin{itemize}
	\tightlist
	\item Análisis sin contacto y no destructivo.
	\item No se suele necesitar preparar la muestra
	\item Sirve para materia orgánica e inorgánica.
	\item Se puede usar para elementos en cualquier estado de la materia
	\item Para conseguir un espectro la exposición de la muestra al láser está entre 10ms y 1s
	
\end{itemize}

En el campo de la geología esta técnica es muy útil debido a la complejidad de la estructura de las rocas, generalmente formadas por varios tipos de minerales, además permite obtener información en profundidad sobre su formación debido a que la espectroscoia Raman es muy sensible al mínimo cambio en la estructura\cite{quora:raman-geology}.

En el área de lenguajes y sistemas informáticos existen colaboraciones en curso entre el grupo de investigación \href{http://admirable-ubu.es/}{ADMIRABLE}\footnote{\url{http://admirable-ubu.es/}} y geóloga Susana Jorge Villar, investigadora de la UBU actualmente adscrita al \href{http://www.cenieh.es/}{CENIEH}\footnote{\url{http://www.cenieh.es/}} en un programa de investigación en geoarqueología\cite{susana-cenieh}, aunque es experta en espectroscopia Raman aplicada a astrobiología y arqueología\cite{susana-ubu}.

Este proyecto en colaboración consiste en un estudio sobre predicción del origen y profundidad de variscita usando los espectros Raman obtenidos de las muestras. De esta colaboración surgieron varias técnicas y algoritmos que se encuentran en proceso de ser publicados en artículos de investigación. Sin embargo estos resultados no son fácilmente accesibles a aquellas personas sin unos altos conocimientos técnicos en informática.

Este proyecto busca integrar parte de esos resultados de investigación existentes en una aplicación web, para facilitar el uso de las técnicas y algoritmos desarrollados a los científicos interesados en espectroscopia Raman, de forma que con un par de clicks sean capaces de avanzar en gran medida en su investigación.

Las acciones que este proyecto busca ofrecer son la de visualización, toma de medidas, procesamiento para eliminar ruido, fallos de calibración, fluorescencia, etc y gestión de experimentos de minería de datos.

Aunque inicial y actualmente el proyecto está enfocado en el caso de la geología, con unas clases prefijados, el proyecto podría evolucionar en siguientes versiones para que las clases sean personalizadas y poder usar la aplicación en más ámbitos de la espectroscopia Raman aparte de la geología.

La aplicación web se encuentra disponible en \url{https://spectra-viewer.nanoapp.io/}. Se proporciona un cuenta con ejemplos cargados para su uso, el usuario es ``tfg.visor.ejemplos@gmail.com'' y la contraseña ``tfg\_visor\_ejemplos''.

\section{Estructura de la memoria}

\section{Materiales entregados}