\capitulo{1}{Introducción}
\label{ch:introduccion}

La espectroscopia Raman es una técnica espectroscópica basada en la dispersión
inelástica de fotones, o dispersión Raman, de la luz monocromática disparada
comúnmente desde un láser~\cite{raman-basics}, generalmente en el rango de la 
luz visible o cercano al infrarrojo o ultravioleta~\cite{wiki:raman-en}. Al 
medir la energía de estos fotones dispersados se obtienen espectros que revelan
información estructural sobre el elemento o material. Esta técnica se explica
más en profundidad en la sección~\ref{def:raman}.

En el campo de la geología esta técnica es muy útil debido a la complejidad en
la composición de las rocas, generalmente formadas por varios tipos de
minerales, que esta técnica es capaz de averiguar. Además permite obtener
información en profundidad sobre su formación debido a que la espectroscopia 
Raman es muy sensible al mínimo cambio en la estructura del 
material~\cite{quora:raman-geology}.

En el área de Lenguajes y Sistemas Informáticos existen colaboraciones en curso
entre el grupo de investigación
\hrefFootnote{http://admirable-ubu.es/}{ADMIRABLE} y la geóloga Susana Jorge
Villar, investigadora de la UBU actualmente adscrita al
\href{http://www.cenieh.es/}{CENIEH}\footnote{\url{http://www.cenieh.es/}} en un
programa de investigación en geoarqueología~\cite{susana-cenieh}, aunque es
experta en espectroscopia Raman aplicada a astrobiología y
arqueología~\cite{susana-ubu}.

Este proyecto en colaboración consiste en un estudio sobre predicción del origen
y profundidad de variscita usando los espectros Raman obtenidos de las muestras.
De esta colaboración surgieron varias técnicas y algoritmos que se encuentran en
proceso de ser publicados en artículos de investigación. Sin embargo, estos
resultados no son fácilmente accesibles a aquellas personas sin unos altos
conocimientos técnicos en informática.

Este proyecto busca integrar en una aplicación web parte de esos resultados de
investigación existentes, para facilitar el uso de las técnicas y algoritmos
desarrollados a los científicos interesados en espectroscopia Raman, de forma
que con un par de acciones con el ratón sean capaces de avanzar en gran medida
en su investigación.

La funcionalidad que este proyecto busca ofrecer son la de visualización, toma
de medidas, procesamiento para eliminar ruido, fallos de calibración,
fluorescencia, etc y gestión de experimentos de minería de datos.

Aunque inicial y actualmente el proyecto está enfocado en el caso de la
geología, con unas clases prefijados, el proyecto podría evolucionar en
siguientes versiones para que las clases sean personalizadas y poder usar la
aplicación en más ámbitos de la espectroscopia Raman aparte de la geología.

La aplicación web se encuentra disponible en 
\url{https://spectra-viewer.nanoapp.io/}.

\section{Materiales entregados}

Junto con la memoria se entregan los siguientes materiales:
\begin{itemize}
	\tightlist	
	\item Aplicación web Visor de espectos.
	\item \textit{Script} de ejecución.
	\item Pruebas unitarias del sistema.
	\item Ficheros de configuración.
	\item Anexos/documentación técnica.
\end{itemize}
En Internet se encuentran alojados los siguientes recursos:
\begin{itemize}
	\tightlist	
	\item \hrefFootnote{https://spectra-viewer.nanoapp.io/}{Dirección de la
		aplicación}.
	\item \hrefFootnote{https://github.com/IvanBeke/TFG-Visor-de-espectros}{Repositorio de
		la aplicación}.
\end{itemize}
