\capitulo{6}{Trabajos relacionados}

Este apartado sería parecido a un estado del arte de una tesis o tesina. En un
trabajo final grado no parece obligada su presencia, aunque se puede dejar a
juicio del tutor el incluir un pequeño resumen comentado de los trabajos y
proyectos ya realizados en el campo del proyecto en curso.

En este apartado se hace un pequeño resumen o mención de otras herramientas o
artículos sobre el tema que puedan estar relacionados con el tema tratado en el
proyecto.

\section{Librerías}
\subsection{Scikit-spectra}
Al principio del proyecto se habló mucho con los tutores de probar y usar esta
librería como referencia o como base del proyecto. Pero después de analizar su
repositorio se vio que llevaba tiempo sin mantenimiento y al instalar y probar
los ejemplos que trae incluidos salían errores por todas partes, por lo que se
dejó de lado. La principal característica de la librería es la visualización y
la construcción de interfaces gráficas mediante \textit{IPython Notebooks} para
su ejecución en navegador\cite{art:skspec}.
