\capitulo{5}{Aspectos relevantes del desarrollo del proyecto}

En este apartado se van a recoger los aspectos más importantes del desarrollo
del proyecto. Desde las decisiones que se tomaron y sus implicaciones,
hasta los numerosos problemas a los que hubo que enfrentarse y cómo se
solucionaron.

\section{Elección del proyecto}

A finales del curso pasado se organizó una charla en la que los profesores iban
a presentar las optativas que daban clase de forma que los alumnos tuviéramos
más fácil elegir asignaturas. En la presentación de la asignatura ``Minería de
Datos'' despertó interés por el tema y se preguntó a José Francisco, por haber
realizado la presentación, sobre TFGs relacionados con el tema.

De los trabajos disponibles este llamó la atención por estar relacionados con
geología y con desarrollo web, además de poder aplicar técnicas minería de datos
en un entorno real de investigación.

\section{Formación}

Para poder realizar el proyecto se necesitaban unos conocimientos no adquiridos
sobre desarrollo web, tanto de la parte de servidor en Flask como la parte del
cliente en HTML, CSS y JavaScript, aunque en menor medida por haberse tocado
algo durante el grado. Como se había hablado con los tutores antes de verano
sobre el proyecto, se dedicó parte a aprender sobre ello. Además de para
aprendizaje, los recursos se han usado también como material de consulta durante
el desarrollo.

\noindent Para la parte del servidor se siguieron los libros y tutoriales:
\begin{itemize}
	\item Flask Web Development\cite{grinberg2014flask}
	\item Explore Flask\cite{exploreflask}
	\item The Flask Mega-Tutorial Legacy (2012)\cite{grinberg-mega-legacy}
	\item The Flask Mega-Tutorial (2017)\cite{grinberg-mega}
\end{itemize}

\noindent Para la parte del cliente se utilizaron principalmente los siguientes
materiales:
\begin{itemize}
	\item MDN Web Docs\cite{mdn}
	\item W3Schools Tutorials\cite{w3schools}
\end{itemize}

A medida que se añadían nuevas herramientas al proyecto, su documentación
oficial también ha sido consultada en varias ocasiones, están disponibles en:
\begin{itemize}
	\item Flask\cite{doc:flask}
	\item Bootstrap\cite{doc:bootstrap}
	\item Nanobox\cite{doc:nanobox}
	\item PyMongo\cite{doc:pymongo}
	\item Dash\cite{doc:dash}
	\item Plotly\cite{doc:plotly}
\end{itemize}

\section{Sistema de usuarios}

Uno de los primeros problemas que se plantearon fue la forma de ofrecer el
sistema de usuarios, para que cada uno pudiera almacenar sus archivos. Las
opciones que se presentaban eran implementar uno desde cero con ayuda de las
extensiones que ofrece Flask o mediante un sistema de terceros, como puede ser
Google.

Implementar el sistema desde cero tenía la ventaja de que los usuarios no tenían
que salir de la página para iniciar sesión y que se había aprendido como hacerlo
en los tutoriales sobre Flask mencionados anteriormente, mientras que el sistema
de terceros no se sabía como hacerlo, pero tenía más ventajas, siendo las
principales no tener que mantener una base de datos de usuarios, no depender de
un sistema de envío de correo electrónico (dio problemas en proyectos
anteriores) y no tener que obligar a los usuarios a crearse otra cuenta al poder
usar una existente.

Al final se decidió usar la autenticación con Google, por estar casi garantizado
que los usuarios van a tener una cuenta existente, la documentación sobre este
aspecto es abundante y era fácil de implementar. Además al iniciar sesión con
este servicio nos permite usar la API de Google para obtener los datos del
usuario necesarios.

\section{Despliegue}

\subsection{Heroku}

\subsection{Problemas}

\subsection{Nanobox como solución}

\section{Procesamiento}

\section{Tecnología y frameworks}

\subsection{Decisiones}

\subsection{Compatibilidad}

\subsection{Cohesión}

\section{Almacenamiento de datos}

\subsection{Inicialmente}

\subsection{Problemas}

\subsection{Migración a MongoDB}

\subsubsection{Primera aproximación}

\subsubsection{Estructura definitiva}
