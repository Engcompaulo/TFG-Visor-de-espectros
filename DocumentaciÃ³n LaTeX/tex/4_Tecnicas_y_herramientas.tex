\capitulo{4}{Técnicas y herramientas}

\section{Librerías de representación}

\subsection{Dash}
Librería en Python que permite crear sitios webs completos para representación de datos. Para ello hace uso de diversas tecnologías, \textit{Flask} para el servidor web, \textit{Plotly} para la representación y \textit{React} para los componentes y actualización.
\subsubsection{Pros}
\begin{itemize}
	\item Gráficos interactivos
	\item Fácil actualización del gráfico en la web mediante \verb|@app.callback|
	\item Integración de elementos HTML para la actualización del gráfico
	\item Uso de la librería \textit{cufflinks} para unir generar una figura directamente de un \textit{DataFrame}
	\item Al ser de los creadores de \textit{Plotly} y usarlo internamente da la posibilidad de usar sus componentes
	\item Al usar \textit{Flask} como servidor tiene acceso a todas sus ventajas
\end{itemize}
\subsubsection{Contras}
\begin{itemize}
	\item El código HTML hay que escribirlo desde el código de Python, esto hace que se complique el mantenimiento
	\item No se pueden reutilizar las plantillas de \textit{Flask}
\end{itemize}

\subsection{Plotly}
Plataforma para representación de datos, dispone de varias librerías para diferentes lenguajes de programación. Representación online y offline.
\subsubsection{Pros}
\begin{itemize}
	\item Gráficos interactivos
	\item Posibilidad de uso con \textit{Flask} y \textit{Jupyter}
\end{itemize}
\subsubsection{Contras}
\begin{itemize}
	\item Para representar en la web hay que hacer uso de dos versiones de la librería, para Python y para JavaScript
	\item La representación online guarda los gráficos generados en una cuenta asociada de la plataforma
	\item La representación offline devuelve el gráfico en Python, pero para representarlo es necesario convertirlo a JSON, enviarlo a la web y que la parte de JS lo represente
	\item La actualización es necesaria hacerla desde el cliente con JS, donde no se dispone de los datos ni de las utilidades de minería de datos
\end{itemize}

\section{Infraestructura}
\subsection{Jupyter Notebook}
Aplicación web que permite la edición y ejecución de código, Python en este caso, en el navegador, donde también se muestran el resultado de la ejecución. Dispone de \textit{widgets} para interactuar con el programa. Se instala localmente.
\subsubsection{Pros}
\begin{itemize}
	\item Fácil subir archivos al servidor en el menú principal
	\item Al no tener que hacer una interfaz web permite centrarse en la programación del código de minería de datos
	\item Los gráficos generados con \textit{Plotly} se representan directamente en el notebook
	\item Posibilidad de usar \url{https://mybinder.org/} para el despliegue
	\item Actualización del gráfico por medio de los \textit{widgets} e \verb|interact|
\end{itemize}
\subsubsection{Contras}
\begin{itemize}
	\item Menos usable e intuitivo
	\item Al estar el código expuesto el cliente podría alterarlo sin querer
	\item \href{http://jupyter-notebook.readthedocs.io/en/latest/public_server.html}{Solo se puede un usuario en servidor público}
\end{itemize}

\subsection{Flask}
Microframework para aplicaciones web en Python. Aunque por si solo \textit{Flask} no sea muy completo, dispone de una gran cantidad de extensiones oficiales y de la comunidad para suplir todas las características de un framework web completo.
\subsubsection{Pros}
\begin{itemize}
	\item Maneja bien la subida de ficheros
	\item Al ser web hay más control sobre lo que puede hacer el usuario y sobre lo que se le presenta, con la finalidad de hacer más usable la aplicación
	\item Reutilización de código HTML mediante plantillas y macros
\end{itemize}
\subsubsection{Contras}
\begin{itemize}
	\item Mucho más trabajo al tener que diseñar y programar la interfaz web
\end{itemize}

\section{Despliegue}
\url{https://www.youtube.com/watch?v=vGphzPLemZE}\\
\url{https://gumroad.com/l/python-deployments}\\
\url{https://www.fullstackpython.com/platform-as-a-service.html}\\
\url{https://www.fullstackpython.com/servers.html}
\subsection{Heroku}
Plataforma como servicio, la forma más fácil de despliegue. Tan escalable como fondo tenga la cartera. \url{https://www.heroku.com/}\\

El almacenamiento no es permanente, hay que usar servicios de terceros y conectarlos mediante plugin.
\subsection{Ngrok}
Túnel seguro desde Internet hasta un servidor local en tu máquina. Dirección aleatoria cada vez que se enciende. \url{https://ngrok.com/}\\

El almacenamiento es permanente porque es el almacenamiento de la máquina.
\subsection{Digital Ocean}
Solo de pago pero de momento está disponible por el pack educacional de GitHub. Tan escalable como fondo tenga la cartera. VPS.\\ \url{https://www.digitalocean.com/}\\ \url{https://pythonprogramming.net/basic-flask-website-tutorial/}\\

Dispone del almacenamiento que ofrece la máquina virtual, es permanente. En caso de que ese espacio sea insuficiente se puede agregar más pagando.
\subsection{Google App Engine, Google Cloud Platform}
Despliegue de Google como plataforma como servicio o VPS. Periodo de prueba gratis y luego tan escalable como fondo tenga la cartera. Funciona con Python 2.7\\
\url{https://cloud.google.com/appengine/docs/standard/python/getting-started/python-standard-env}\\
\url{https://cloud.google.com/appengine/docs/standard/python/tools/uploadinganapp}\\
\url{https://cloud.google.com/python/getting-started/hello-world}\\
\url{https://cloud.google.com/appengine/docs/flexible/python/quickstart}\\

Ofrece formas de almacenamiento de Google como Cloud Storage para ficheros, caso que nos interesa, hay que pagar por ellas.
\subsection{Open Shitf}
Plataforma como servicio. Plan básico gratis y plan profesional de pago, tan escalable como fondo tenga la cartera.\\
\url{https://www.openshift.com/}\\
\url{https://blog.openshift.com/beginners-guide-to-writing-flask-apps-on-openshift/}\\
\url{https://blog.openshift.com/how-to-install-and-configure-a-python-flask-dev-environment-deploy-to-openshift/}\\

No ofrece almacenamiento por defecto, lo ofrece por medio de lo que llaman ``PersistentVolume'', ofrecen una API para comunicarse con ello.
\subsection{PythonAnywhere} 
Plataforma como servicio especializada en Python. Varios planes, a mejor plan más caro. El plan más básico es gratis.\\
\url{https://www.pythonanywhere.com/}\\
\url{https://www.youtube.com/watch?v=M-QRwEEZ9-8}\\
Ofrecen almacenamiento de serie pero muy limitado y de pago.
\subsection{AWS Elastic Beanstalk,  AWS CodeStar}
Solución en la nube de Amazon. VPS. Tan escalable como fondo tenga la cartera.\\
\url{https://aws.amazon.com/es/elasticbeanstalk/}\\
\url{https://aws.amazon.com/es/codestar/}\\

Al igual que en el caso de Google, ofrece almacenamiento persistente con sus servicios, Amazon S3, los cuales hay que pagar.
\subsection{AWS Lambda, Zappa}
Zappa es un capa por encima de AWS Lambda para desplegar en modo \textit{serverless}. AWS Lambda se ocupa del escalado y Zappa del despliegue.\\
\url{https://github.com/Miserlou/Zappa}\\

Solo almacenamiento temporal.
\subsection{Docker}
Despliegue en contenedores.\\
\url{https://www.docker.com/}\\

De serie no tiene almacenamiento persistente pero es capaz de ofrecerlo mediante ``storage drivers''. Requiere bastante configuración.
\subsection{Azure}
Solución en la nube de Microsoft. VPS. Tan escalable como fondo tenga la cartera.\\
\url{https://azure.microsoft.com/es-es/}\\
\url{https://docs.microsoft.com/en-us/azure/app-service/app-service-web-get-started-python}\\

Sí ofrece almacenamiento persistente.
\subsection{Nanobox}
Solución interesante, combina los contenedores de Docker con despliegue en la nube y lo automatiza. De momento compatibilidad con Digital Ocean, Amazon y Linode, Google, Joyent y Azure en camino. Plan básico gratis, el resto de precios son flexibles. También en local.\\
\url{https://nanobox.io/}\\
\url{https://github.com/nanobox-io/nanobox}\\

Ofrece almacenamiento persistente, hay que configurar las rutas que van a ser persistentes en el fichero de configuración. Cada despliegue el almacenamiento que se haya usado so borra. Depende del almacenamiento que esté disponible en el servicio escogido para almacenar.