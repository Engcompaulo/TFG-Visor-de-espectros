\capitulo{4}{Técnicas y herramientas}

\section{Técnicas}

\subsection{Scrum}

Scrum es una metodología de desarrollo ágil basada en los principios del
manifiesto ágil\footnote{\url{http://agilemanifesto.org/}}. Esta metodología
esta pensada para equipos por lo que ha sido adaptada para el desarrollo de este
proyecto.

Scrum consiste en ciclos de trabajo iterativos denominados \textit{sprints}, con
duración de una semana generalmente, en los que al terminar se entrega un
producto funcional. Al final de cada \textit{sprint} se lleva a cabo una reunión
con los tutores para comentar el desarrollo del \textit{sprint}, enseñar los
avances y planear el siguiente ciclo de desarrollo.

El uso de esta metodología permite la entrega de un producto de forma
incremental, además de soportar cambios en los requisitos durante el desarrollo
teniendo en cuenta el \textit{feedback} del usuario, consiguiendo al final un
producto de más calidad y valor para el usuario.

\subsection{Kanban}

Kanban se refiere a un sistema de organización visual de tareas, representadas
como tarjetas, organizadas en diferentes tableros según el estado de desarrollo
en que se encuentren.

Este sistema encaja perfectamente con la metodología Scrum, organizando las
tareas de cada \textit{sprint} para una mayor planificación. Esto permite
centrarse en las tareas que están desarrollo, mantener un registro de las tareas
terminadas y que no se pierdan tareas que están por hacer.

\section{Control de versiones y gestión del proyecto}

\subsection{Git}

Git es un sistema de control de versiones distribuido. No se han considerado
otras alternativas al ser un sistema ya conocido. Actualmente puede ser
considerado el sistema con más usuarios y de más fama.

\subsection{Repositorio remoto}

\begin{itemize}
	\tightlist
	\item Herramientas consideradas: \hrefFootnote{https://github.com/}{GitHub},
	\hrefFootnote{https://bitbucket.org}{Bitbucket},
	\hrefFootnote{https://about.gitlab.com/}{GitLab}
	\item Herramienta escogida: GitHub
\end{itemize}

GitHub es un servicio de alojamiento de repositorios de código basado en
\textit{Git}.

Esta plataforma se ha utilizado para alojar el proyecto, gestionar las tareas
mediante \textit{Issues} y planificar \textit{sprints} mediante
\textit{Milestones}.

La principal razón para escoger esta plataforma es la experiencia previa además
de haber trabajado en ella en asignaturas del grado.

\subsection{Tablero kanban}

\begin{itemize}
	\tightlist
	\item Herramientas consideradas:
	\hrefFootnote{https://www.zenhub.com/}{ZenHub},
	\hrefFootnote{https://www.gitkraken.com/glo}{GitKraken Glo}
	\item Herramienta escogida: ZenHub
\end{itemize}

ZenHub es una herramienta que ofrece tableros kanban para organizar las tareas
de un proyecto. Esta herramienta se integra dentro de GitHub permitiendo tener
toda la gestión centralizada en un mismo lugar.

Las \textit{Issues} de GitHub se representan como tarjetas que organizar en los
tableros. También ofrece la creación de gráficos \textit{burndown} para
visualizar el desarrollo del \textit{sprint}. 

Al igual que en el caso anterior, se ha escogido por tener experiencia previa en
asignaturas del grado.

\subsection{Interfaz gráfico}

\begin{itemize}
	\tightlist
	\item Herramientas consideradas:
	\hrefFootnote{https://www.sourcetreeapp.com/}{Sourcetree},
	\hrefFootnote{https://www.gitkraken.com/}{GitKraken},
	\hrefFootnote{https://desktop.github.com/}{GitHub Desktop}
	\item Herramienta escogida: GitKraken
\end{itemize}

GitKraken es un \textit{software} que proporciona una interfaz gráfica al
sistema \textit{Git} permitiendo realizar todas sus acciones de una forma más
sencilla y visual. Se puede enlazar con los repositorios remotos principales y
está disponible para Windows, Mac y Linux.

Aunque se ha trabajado con las tres herramientas mencionadas, se ha elegido por
ser actualmente la única de ellas disponible para Linux. La versión Pro de este
programa está incluida dentro del
\hrefFootnote{https://education.github.com/pack}{Student Developer Pack}.

\section{Herramientas}

\subsection{Flask}

Una de las condiciones iniciales del proyecto es el uso de Python para
desarrollar el servidor web con el \textit{framework}
\hrefFootnote{http://flask.pocoo.org/}{Flask}.

Este \textit{framework} es bastante fácil de utilizar y no impone ninguna
limitación respecto a estructura del proyecto ni a los componentes que usar
durante el desarrollo~\cite{grinberg2014flask}.

Debido a su simplicidad tiene bastantes carencias respecto a funcionalidades
avanzadas pero gracias a los complementos que la gran comunidad que se ha
formado a su alrededor ha desarrollado deja de ser un problema.

\subsection{Librería de representación}

\begin{itemize}
	\tightlist
	\item Herramientas consideradas: \hrefFootnote{https://plot.ly/}{Plotly},
	\hrefFootnote{https://plot.ly/products/dash/}{Dash},
	\hrefFootnote{https://bokeh.pydata.org/en/latest/}{Bokeh}
	\item Herramienta escogida: Dash
\end{itemize}

Dash es un \textit{framework} desarrollo web construido sobre Flask, ReactJS y
Plotly para la creación de aplicaciones web centradas en visualización de datos.
Su filosofía es escribir todo el código desde Python.

Permite asociar funciones a componentes de la interfaz que son ejecutadas cuando
se realiza en evento concreto sobre ellos. Esto es una gran ventaja para el
proyecto, ya que facilita en gran medida la actualización de los gráficos al
aplicar opciones de preprocesamiento.

\subsection{Scikit-learn}

\hrefFootnote{http://scikit-learn.org}{Scikit-learn}~\cite{art:scikit-learn} es
una librería de aprendizaje automático en Python. Contiene una amplia cantidad
de algoritmos de clasificación, regresión y \textit{clustering}.

Esta librería proporciona los modelos usados en el proyecto, así como las
funciones para calcular las estadísticas de los modelos entrenados.

\subsection{MongoDB}

\hrefFootnote{https://www.mongodb.com/}{MongoDB} es una base de datos NoSQL de
tipo documental. permite almacenar datos cuya estructura sea similar a JSON. Las
consultas a esta base de datos son muy sencillas de realizar y la respuesta es
bastante rápida.

Se ha escogido esta plataforma por la simplicidad con las que se pueden insertar
y recuperar \textit{DataFrames} de \href{https://pandas.pydata.org/}{pandas},
una estructura de datos similar a una tabla que puede exportarse a formato JSON.

Aunque al principio del desarrollo no se requería el uso de una base de datos,
en la sección~\ref{sec:migrate-mongo} se explican las razones que motivaron el
cambio.

\subsection{Entorno de desarrollo integrado (IDE}

\begin{itemize}
	\tightlist
	\item Herramientas consideradas:
	\hrefFootnote{https://www.jetbrains.com/pycharm/}{PyCharm},
	\hrefFootnote{https://atom.io/}{Atom},
	\hrefFootnote{https://code.visualstudio.com/}{VS Code}
	\item Herramienta escogida: PyCharm
\end{itemize}

PyCharm es un IDE desarrollado por la empresa JetBrains para el lenguaje Python.
Existen dos ediciones, \textit{Community} y \textit{Professional}. Gracias al
programa \hrefFootnote{https://www.jetbrains.com/student/}{For Students} de
JetBrains se está usando la versión \textit{Professional} en el proyecto, aunque
la \textit{Community} serviría perfectamente.

Ofrece soporte para Flask de serie y soporte para desarrollo web mediante
complementos, razones principales de la elección.

\subsection{Documentación}

\begin{itemize}
	\tightlist
	\item Herramientas consideradas: \LaTeX, Microsoft Office, Libre Office
	\item Herramienta escogida: \LaTeX
\end{itemize}

\LaTeX es un lenguaje orientado a la creación de documentos que presenten alta
calidad tipográfica.

Aunque su curva de aprendizaje es elevada, posee grande ventajas como no tener
que preocuparse del formato del documento o que los textos no pierdan calidad al
ser ampliados o impresos.

\subsection{Calidad del código}

\begin{itemize}
	\tightlist
	\item Herramientas consideradas:
	\hrefFootnote{https://www.codacy.com/}{Codacy},
	\hrefFootnote{https://codeclimate.com/}{Code Climate}
	\item Herramienta escogida: Codacy
\end{itemize}

Codacy es una herramienta de revisión automática de código. Se integra con
GitHub y analiza los proyectos que se hayan seleccionado cuando se produce un
\textit{push} al repositorio.

Esta herramienta detecta problemas de seguridad, de estilo de código,
complejidad, código duplicado y código duplicado. Con estos datos se elaboran
gráficos históricos de la evolución del proyecto.

Después de los análisis se asigna al proyecto una certificación según calidad
basada en letras. Durante el desarrollo se ha intentado mantener una
Certificación A, la más alta. 

El análisis del proyecto se puede encontrar
\url{https://app.codacy.com/project/IvanBeke/TFG-Visor-de-espectros/dashboard}

\subsection{Monitorización de dependencias}

\subsubsection{Requires.io}

Esta herramienta analiza las dependencias del proyecto en busca de nuevas
versiones e informa cuando una ha sacado nueva versión. También es capaz de
mostrar la licencia de software que posee la dependencia y de ofrecer un enlace
a su repositorio.

El análisis de dependencias del proyecto está disponible en
\url{https://requires.io/github/IvanBeke/TFG-Visor-de-espectros/requirements/?branch=master}.

\subsubsection{Snyk}

Esta herramienta analiza las dependencias de nuestro proyecto para informar al
usuario de vulnerabilidades de las mismas, de tal forma que puedan preparar
medidas contra estos fallos.

El informe de vulnerabilidades está disponible en
\url{https://snyk.io/test/github/IvanBeke/TFG-Visor-de-espectros?targetFile=requirements.txt}.

\section{Despliegue}

\subsection{Alojamienta}

\begin{itemize}
	\tightlist
	\item Herramientas consideradas:
	\hrefFootnote{https://www.heroku.com/}{Heroku},
	\hrefFootnote{https://www.digitalocean.com/}{Digital Ocean},
	\hrefFootnote{https://cloud.google.com/}{Google Cloud},
	\hrefFootnote{https://www.openshift.com/}{Open Shift},
	\hrefFootnote{https://www.pythonanywhere.com/}{PythonAnywhere},
	\hrefFootnote{https://aws.amazon.com/es/}{Amazon Web Services}
	\item Herramienta escogida: Digital Ocean
\end{itemize}

Digital Ocean es un proveedor de servicios de computación en la nube. Ofrece
servidores virtuales privados para usarlos como se quiera dentro de unas
limitaciones en procesador, memoria y tamaño de disco.

Aunque al principio se usó Heroku se cambió la plataforma escogida por los
motivos explicados en la sección~\ref{sec:despliegue}. Gracias al
\hrefFootnote{https://education.github.com/pack}{Student Developer Pack} se pudo
hacer uso de esta plataforma por ofrecer un cupón de 50\$.

\subsection{Nanobox}

\hrefFootnote{https://nanobox.io/}{Nanobox} es una plataforma con el objetivo de
facilitar la tarea del despliegue de páginas web sin que haya que preocuparse de
la configuración del servidor ni su infraestructura.

Para ello se enlaza con proveedores de servicios de computación de la nube y
mediante una herramienta en línea de comandos e instrucciones simples se ocupa
de todo el proceso para tener la aplicación web lista.
