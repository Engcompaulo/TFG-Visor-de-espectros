\documentclass[a4paper,12pt,twoside]{memoir}

% Castellano
\usepackage[spanish,es-tabla]{babel}
\selectlanguage{spanish}
\usepackage[utf8]{inputenc}
\usepackage[T1]{fontenc}
\usepackage{lmodern} % Scalable font
\usepackage{microtype}
\usepackage{placeins}
\usepackage{xspace}

\RequirePackage{booktabs}
\RequirePackage[table]{xcolor}
\RequirePackage{xtab}
\RequirePackage{multirow}

% Links
\usepackage[colorlinks]{hyperref}
\hypersetup{
	allcolors = {red}
}

% Ecuaciones
\usepackage{amsmath}

% Rutas de fichero / paquete
\newcommand{\ruta}[1]{{\sffamily #1}}

% Párrafos
\nonzeroparskip


% Imagenes
\usepackage{graphicx}
\newcommand{\imagen}[2]{
	\begin{figure}[!h]
		\centering
		\includegraphics[width=0.9\textwidth]{#1}
		\caption{#2}\label{fig:#1}
	\end{figure}
	\FloatBarrier
}

\newcommand{\imagenflotante}[2]{
	\begin{figure}%[!h]
		\centering
		\includegraphics[width=0.9\textwidth]{#1}
		\caption{#2}\label{fig:#1}
	\end{figure}
}



% El comando \figura nos permite insertar figuras comodamente, y utilizando
% siempre el mismo formato. Los parametros son:
% 1 -> Porcentaje del ancho de página que ocupará la figura (de 0 a 1)
% 2 --> Fichero de la imagen
% 3 --> Texto a pie de imagen
% 4 --> Etiqueta (label) para referencias
% 5 --> Opciones que queramos pasarle al \includegraphics
% 6 --> Opciones de posicionamiento a pasarle a \begin{figure}
\newcommand{\figuraConPosicion}[6]{%
	\setlength{\anchoFloat}{#1\textwidth}%
	\addtolength{\anchoFloat}{-4\fboxsep}%
	\setlength{\anchoFigura}{\anchoFloat}%
	\begin{figure}[#6]
		\begin{center}%
			\Ovalbox{%
				\begin{minipage}{\anchoFloat}%
					\begin{center}%
						\includegraphics[width=\anchoFigura,#5]{#2}%
						\caption{#3}%
						\label{#4}%
					\end{center}%
				\end{minipage}
			}%
		\end{center}%
	\end{figure}%
}

%
% Comando para incluir imágenes en formato apaisado (sin marco).
\newcommand{\figuraApaisadaSinMarco}[5]{%
	\begin{figure}%
		\begin{center}%
			\includegraphics[angle=90,height=#1\textheight,#5]{#2}%
			\caption{#3}%
			\label{#4}%
		\end{center}%
	\end{figure}%
}
% Para las tablas
\newcommand{\otoprule}{\midrule [\heavyrulewidth]}
%
% Nuevo comando para tablas pequeñas (menos de una página).
\newcommand{\tablaSmall}[5]{%
	\begin{table}
		\begin{center}
			\rowcolors {2}{gray!35}{}
			\begin{tabular}{#2}
				\toprule
				#4
				\otoprule
				#5
				\bottomrule
			\end{tabular}
			\caption{#1}
			\label{tabla:#3}
		\end{center}
	\end{table}
}

%
% Nuevo comando para tablas pequeñas (menos de una página).
\newcommand{\tablaSmallSinColores}[5]{%
	\begin{table}[H]
		\begin{center}
			\begin{tabular}{#2}
				\toprule
				#4
				\otoprule
				#5
				\bottomrule
			\end{tabular}
			\caption{#1}
			\label{tabla:#3}
		\end{center}
	\end{table}
}

\newcommand{\tablaApaisadaSmall}[5]{%
	\begin{landscape}
		\begin{table}
			\begin{center}
				\rowcolors {2}{gray!35}{}
				\begin{tabular}{#2}
					\toprule
					#4
					\otoprule
					#5
					\bottomrule
				\end{tabular}
				\caption{#1}
				\label{tabla:#3}
			\end{center}
		\end{table}
	\end{landscape}
}

%
% Nuevo comando para tablas grandes con cabecera y filas alternas coloreadas en
%gris.
\newcommand{\tabla}[6]{%
	\begin{center}
		\tablefirsthead{
			\toprule
			#5
			\otoprule
		}
		\tablehead{
			\multicolumn{#3}{l}{\small\sl continúa desde la página anterior}\\
			\toprule
			#5
			\otoprule
		}
		\tabletail{
			\hline
			\multicolumn{#3}{r}{\small\sl continúa en la página siguiente}\\
		}
		\tablelasttail{
			\hline
		}
		\bottomcaption{#1}
		\rowcolors {2}{gray!35}{}
		\begin{xtabular}{#2}
			#6
			\bottomrule
		\end{xtabular}
		\label{tabla:#4}
	\end{center}
}

%
% Nuevo comando para tablas grandes con cabecera.
\newcommand{\tablaSinColores}[6]{%
	\begin{center}
		\tablefirsthead{
			\toprule
			#5
			\otoprule
		}
		\tablehead{
			\multicolumn{#3}{l}{\small\sl continúa desde la página anterior}\\
			\toprule
			#5
			\otoprule
		}
		\tabletail{
			\hline
			\multicolumn{#3}{r}{\small\sl continúa en la página siguiente}\\
		}
		\tablelasttail{
			\hline
		}
		\bottomcaption{#1}
		\begin{xtabular}{#2}
			#6
			\bottomrule
		\end{xtabular}
		\label{tabla:#4}
	\end{center}
}

%
% Nuevo comando para tablas grandes sin cabecera.
\newcommand{\tablaSinCabecera}[5]{%
	\begin{center}
		\tablefirsthead{
			\toprule
		}
		\tablehead{
			\multicolumn{#3}{l}{\small\sl continúa desde la página anterior}\\
			\hline
		}
		\tabletail{
			\hline
			\multicolumn{#3}{r}{\small\sl continúa en la página siguiente}\\
		}
		\tablelasttail{
			\hline
		}
		\bottomcaption{#1}
		\begin{xtabular}{#2}
			#5
			\bottomrule
		\end{xtabular}
		\label{tabla:#4}
	\end{center}
}



\definecolor{cgoLight}{HTML}{EEEEEE}
\definecolor{cgoExtralight}{HTML}{FFFFFF}

%
% Nuevo comando para tablas grandes sin cabecera.
\newcommand{\tablaSinCabeceraConBandas}[5]{%
	\begin{center}
		\tablefirsthead{
			\toprule
		}
		\tablehead{
			\multicolumn{#3}{l}{\small\sl continúa desde la página anterior}\\
			\hline
		}
		\tabletail{
			\hline
			\multicolumn{#3}{r}{\small\sl continúa en la página siguiente}\\
		}
		\tablelasttail{
			\hline
		}
		\bottomcaption{#1}
		\rowcolors[]{1}{cgoExtralight}{cgoLight}
		
		\begin{xtabular}{#2}
			#5
			\bottomrule
		\end{xtabular}
		\label{tabla:#4}
	\end{center}
}

\newcommand{\hrefFootnote}[2]{%
	\href{#1}{#2}\footnote{\url{#1}}\xspace
}


\newcommand{\code}[1]{\texttt{#1}}


\graphicspath{ {./img/} }

% Capítulos
\chapterstyle{bianchi}
\newcommand{\capitulo}[2]{
	\setcounter{chapter}{#1}
	\setcounter{section}{0}
	\chapter*{#2}
	\addcontentsline{toc}{chapter}{#2}
	\markboth{#2}{#2}
}

% Apéndices
\renewcommand{\appendixname}{Apéndice}
\renewcommand*\cftappendixname{\appendixname}

\newcommand{\apendice}[1]{
	%\renewcommand{\thechapter}{A}
	\chapter{#1}
}

\renewcommand*\cftappendixname{\appendixname\ }

% Formato de portada
\makeatletter
\usepackage{xcolor}
\newcommand{\tutor}[1]{\def\@tutor{#1}}
\newcommand{\cotutor}[1]{\def\@cotutor{#1}}
\newcommand{\course}[1]{\def\@course{#1}}
\definecolor{cpardoBox}{HTML}{E6E6FF}
\def\maketitle{
	\null
	\thispagestyle{empty}
	% Cabecera ----------------
	\noindent\includegraphics[width=\textwidth]{cabecera}\vspace{1cm}%
	\vfill
	% Título proyecto y escudo informática ----------------
	\colorbox{cpardoBox}{%
		\begin{minipage}{.8\textwidth}
			\vspace{.5cm}\Large
			\begin{center}
				\textbf{TFG del Grado en Ingeniería Informática}\vspace{.6cm}\\
				\textbf{\LARGE\@title{}}
			\end{center}
			\vspace{.2cm}
		\end{minipage}
		
	}%
	\hfill\begin{minipage}{.20\textwidth}
		\includegraphics[width=\textwidth]{escudoInfor}
	\end{minipage}
	\vfill
	% Datos de alumno, curso y tutores ------------------
	\begin{center}%
		{%
			\noindent\LARGE
			Presentado por \@author{}\\ 
			en Universidad de Burgos --- \@date{}\\
			Tutor: \@tutor{}\\
			Cotutor: \@cotutor{}\\
		}%
	\end{center}%
	\null
	\cleardoublepage
}
\makeatother

\newcommand{\nombre}{Iván Iglesias Cuesta} %%% cambio de comando

% Datos de portada
\title{Visor de espectros}
\author{\nombre}
\tutor{Dr. José Francisco Díez Pastor}
\cotutor{Dr. César Ignacio García Osorio}
\date{\today}

\begin{document}
	
	\maketitle
	
	
	\newpage\null\thispagestyle{empty}\newpage
	
	
	
	%%%%%%%%%%%%%%%%%%%%%%%%%%%%%%%%%%%%%%%%%%%%%%%%%%%%%%%%%%%%%%%%%%%%%%%%%%%%%%%%%%%%%%%%
	\thispagestyle{empty}
	
	
	\noindent\includegraphics[width=\textwidth]{cabecera}\vspace{1cm}
	
	\noindent D. José Francisco Díez Pastor y D. César Ignacio García Osorio, profesores del
	departamento de Ingeniería Civil, área de Lenguajes y Sistemas Informáticos
	
	\noindent Exponen:
	
	\noindent Que el alumno D. \nombre, con DNI 45573756S, ha realizado el Trabajo final
	de Grado en Ingeniería Informática titulado ``Visor de espectros''. 
	
	\noindent Y que dicho trabajo ha sido realizado por el alumno bajo la dirección
	de los que suscriben, en virtud de lo cual se autoriza su presentación y defensa.
	
	\begin{center} %\large
		En Burgos, {\large \today}
	\end{center}
	
	\vfill\vfill\vfill
	
	% Author and supervisor
	\begin{minipage}{0.45\textwidth}
		\begin{flushleft} %\large
			Vº. Bº. del Tutor:\\[2cm]
			D. Jośe Francisco Díez Pastor
		\end{flushleft}
	\end{minipage}
	\hfill
	\begin{minipage}{0.45\textwidth}
		\begin{flushleft} %\large
			Vº. Bº. del co-tutor:\\[2cm]
			D. César Ignacio García Osorio
		\end{flushleft}
	\end{minipage}
	\hfill
	
	\vfill
	
	% para casos con solo un tutor comentar lo anterior
	% y descomentar lo siguiente
	%Vº. Bº. del Tutor:\\[2cm]
	%D. nombre tutor
	
	
	\newpage\null\thispagestyle{empty}\newpage
	
	
	
	
	\frontmatter
	
	% Abstract en castellano
	\renewcommand*\abstractname{Resumen}
	\begin{abstract}
		La espectroscopia Raman es una técnica de análisis no destructivo usada para
		conocer la estructura y composición de un material o elemento. En el campo de
		la
		geología el uso de esta técnica es común para determinar la composición,
		origen
		o profundidad de muestras de minerales extraídos.
		
		Actualmente, se están empezando a usar técnicas de minería de datos para la
		identificación de estos espectros. Aunque existen herramientas que son capaces
		de visualizar y aplicar ciertas operaciones sobre los espectros, no existe un
		software específico que facilite la aplicación de técnicas de minería de datos
		sobre los espectros.
		
		Este proyecto se realiza en colaboración con una investigadora en geología que
		usa esta técnica para el análisis de un mineral en concreto llamado variscita.
		Este proyecto parte de unas técnicas y algoritmos desarrollados en
		colaboraciones previas.
		
		El desarrollo de este proyecto busca desarrollar una aplicación web que
		permita
		cargar los espectros, visualizarlos y procesarlos, así como aplicar técnicas
		de
		minería de datos para construir modelos de clasificación para nuevas muestras.
	\end{abstract}
	
	\renewcommand*\abstractname{Descriptores}
	\begin{abstract}
		Aprendizaje automático, minería de datos, procesamiento de espectros,
		variscita,
		espectroscopia Raman, aplicación web.
	\end{abstract}
	
	\clearpage
	
	% Abstract en inglés
	\renewcommand*\abstractname{Abstract}
	\begin{abstract}
		Raman spectroscopy is a non-destructive analysis technique used to know the
		structure and composition of a material or element. In the field of geology
		the
		use of this technique is common to determine the composition, origin or depth
		of
		extracted mineral samples.
		
		Currently, data mining techniques are beginning to be used for the
		identification
		of these spectra. Although there are tools that are able to visualize and
		apply
		certain operations on the spectra, there is no specific software that
		facilitates the application of data mining techniques on the spectra.
		
		This project is carried out in collaboration with a researcher in geology who
		uses this technique for the analysis of a particular mineral called variscite.
		This project is based on techniques and algorithms developed in previous
		collaborations.
		
		The development of this project seeks to develop a web application that allows
		to load the spectra, visualize and process them, as well as applying data
		mining
		techniques to build classification models for new samples.
	\end{abstract}
	
	\renewcommand*\abstractname{Keywords}
	\begin{abstract}
		Machine learning, data mining, spectra processing, variscite, Raman
		espectroscopy, web application.
	\end{abstract}
	
	\clearpage
	
	% Indices
	\tableofcontents
	
	\clearpage
	
	\listoffigures
	
	\clearpage
	
	\mainmatter
	\capitulo{1}{Introducción}

La espectroscopia Raman es una técnica espectroscópica basada en la dispersión
inelástica de fotones, o dispersión Raman, de la luz monocromática disparada
comúnmente desde un láser\cite{raman-basics}, generalmente en el rango de la luz
visible o cercano al infrarrojo o ultravioleta\cite{wiki:raman-en}.

La dispersión de fotones se refiere a como los fotones lanzados por el láser
rebotan, mejor dicho, son absorbidos y vueltos a emitir. Los fotones se pueden
dispersar de forma elástica (Rayleigh) o inelástica (Raman).

En la primera forma los fotones absorbidos son emitidos de vuelta igual que
fueron absorbidos, la gran mayoría de ellos, pero una pequeña se emiten
cambiados, con un pequeño cambio en la su energía, ya sea a más o menos (ver
Figura~\ref{fig:dispersion-raman}).

Este cambio de energía varía según el material o elemento contra el que impacten
los fotones, revelando ahí la estructura o composición y abriendo un amplio
abanico de aplicaciones para esta
técnica\cite{what-is-raman,wiki:raman-scatter}. Con la dispersión Raman
capturada se obtienen los espectros Raman.

\imagen{dispersion-raman}{Representración de la dispersión de
	fotones\cite{what-is-raman}}

Con lo explicado anteriormente se puede ver que esta técnica tiene varias
ventajas, entre las que destacan\cite{what-is-raman}:
\begin{itemize}
	\tightlist
	\item Análisis sin contacto y no destructivo.
	\item No se suele necesitar preparar la muestra
	\item Sirve para materia orgánica e inorgánica.
	\item Se puede usar para elementos en cualquier estado de la materia
	\item Para conseguir un espectro la exposición de la muestra al láser está
	entre 10ms y 1s
	
\end{itemize}

En el campo de la geología esta técnica es muy útil debido a la complejidad de
la estructura de las rocas, generalmente formadas por varios tipos de minerales,
además permite obtener información en profundidad sobre su formación debido a
que la espectroscopia Raman es muy sensible al mínimo cambio en la
estructura\cite{quora:raman-geology}.

En el área de lenguajes y sistemas informáticos existen colaboraciones en curso
entre el grupo de investigación
\href{http://admirable-ubu.es/}{ADMIRABLE}\footnote{\url{http://admirable-ubu.es/}}
y geóloga Susana Jorge Villar, investigadora de la UBU actualmente adscrita al
\href{http://www.cenieh.es/}{CENIEH}\footnote{\url{http://www.cenieh.es/}} en un
programa de investigación en geoarqueología\cite{susana-cenieh}, aunque es
experta en espectroscopia Raman aplicada a astrobiología y
arqueología\cite{susana-ubu}.

Este proyecto en colaboración consiste en un estudio sobre predicción del origen
y profundidad de variscita usando los espectros Raman obtenidos de las muestras.
De esta colaboración surgieron varias técnicas y algoritmos que se encuentran en
proceso de ser publicados en artículos de investigación. Sin embargo estos
resultados no son fácilmente accesibles a aquellas personas sin unos altos
conocimientos técnicos en informática.

Este proyecto busca integrar parte de esos resultados de investigación
existentes en una aplicación web, para facilitar el uso de las técnicas y
algoritmos desarrollados a los científicos interesados en espectroscopia Raman,
de forma que con un par de clicks sean capaces de avanzar en gran medida en su
investigación.

Las acciones que este proyecto busca ofrecer son la de visualización, toma de
medidas, procesamiento para eliminar ruido, fallos de calibración,
fluorescencia, etc y gestión de experimentos de minería de datos.

Aunque inicial y actualmente el proyecto está enfocado en el caso de la
geología, con unas clases prefijados, el proyecto podría evolucionar en
siguientes versiones para que las clases sean personalizadas y poder usar la
aplicación en más ámbitos de la espectroscopia Raman aparte de la geología.

La aplicación web se encuentra disponible en
\url{https://spectra-viewer.nanoapp.io/}. Se proporciona un cuenta con ejemplos
cargados para su uso, el usuario es ``tfg.visor.ejemplos@gmail.com'' y la
contraseña ``tfg\_visor\_ejemplos''.

\section{Estructura de la memoria}

\section{Materiales entregados}



	\capitulo{2}{Objetivos del proyecto}

Este apartado explica de forma precisa y concisa cuales son los objetivos que se
persiguen con la realización del proyecto. Se puede distinguir entre los
objetivos marcados por los requisitos del software a construir y los objetivos
de carácter técnico que plantea a la hora de llevar a la práctica el proyecto.

En esta sección se listan los objetivos que persigue la realización del
proyecto, tanto 

\subsection{Objetivos principales}
\begin{itemize}
	\tightlist
	\item Desarrollar una aplicación que proporcione las opciones de procesamiento
	más comunes.
	\item Que la aplicación final sea útil para la investigadora.
	\item La aplicación tiene que ser usable e intuitiva.
	\item Que la herramienta se consiga desplegar.
	\item Ofrecer control de usuarios para que cada uno pueda almacenar sus
	archivos.
	\item Ofrecer un sistema de aprendizaje automático para ayudar en la
	clasificación de futuras muestras de espectros.
\end{itemize}

\subsection{Objetivos técnicos}
\begin{itemize}
	\tightlist
	\item Que la aplicación sea fácil de mantener.
	\item Utilizar git como sistema de control de versiones junto con GitHub para
	el repositorio remoto.
	\item Utilizar una metodología ágil, Scrum, para el desarrollo.
	\item Utilizar un sistema kanban para la gestión de tareas.
	\item Utilizar un sistema de revisión automática de código para asegurar su
	calidad.
	\item Utilizar un sistema de integración continua.
\end{itemize}

\subsection{Objetivos personales}
\begin{itemize}
	\tightlist
	\item Ampliar los conocimientos sobre desarrollo web a partir de los obtenidos
	durante el grado.
	\item Ampliar y profundizar conocimientos sobre Python, especialmente en
	desarrollo web y aprendizaje automático.
	\item Aprender a usar técnicas de aprendizaje automático en un entorno de
	investigación real.
	\item Desarrollar el proyecto de la forma más profesional posible.
\end{itemize}

	\capitulo{3}{Conceptos teóricos}

\section{Espectroscopia Raman}\label{def:raman}
La espectroscopia Raman hace uso del fenómeno conocido como dispersión
inelástica de fotones para obtener gráficas que definen la estructura y
composición de un material o elemento.

Este fenómeno se refiere a como los fotones rebotan, o mejor dicho, son
absorbidos y vueltos a emitir. Los fotones se pueden
dispersar de forma elástica (Rayleigh) o inelástica (Raman).

En la primera forma los fotones absorbidos son emitidos de vuelta igual que
fueron absorbidos, la gran mayoría de ellos, pero una pequeña cantidad se emiten
cambiados, con una pequeña disminución o aumento de sus energía (ver
figura~\ref{fig:dispersion-raman}).

Este cambio de energía varía según el material o elemento contra el que impacten
los fotones, revelando ahí la estructura o composición y abriendo un amplio
abanico de aplicaciones para esta
técnica~\cite{what-is-raman,wiki:raman-scatter}. Con la dispersión Raman
capturada se obtienen los espectros Raman.

\imagen{dispersion-raman}{Representración de la dispersión de
	fotones~\cite{what-is-raman}}

Con lo explicado anteriormente se puede ver que esta técnica tiene varias
ventajas, entre las que destacan~\cite{what-is-raman}:
\begin{itemize}
	\tightlist
	\item Análisis sin contacto y no destructivo.
	\item No se suele necesitar preparar la muestra
	\item Sirve para materia orgánica e inorgánica.
	\item Se puede usar para elementos en cualquier estado de la materia
	\item Para conseguir un espectro la exposición de la muestra al láser está
	entre 10ms y 1s.
\end{itemize}

\section{Visualización de datos}

Técnicas usadas con el propósito de presentar datos o información mediante
gráficos, como puntos, líneas o barras. Está considerado uno de los pasos dentro
del análisis de datos o ciencia de datos~\cite{wiki:dataviz}.

Su objetivo principal es el de comunicas información de forma clara y eficiente,
sin significar esto que para que un gráfico sea funcional tenga que parecer
aburrido ni que tenga que ser extremadamente sofisticado para resultar
agradable~\cite{friedman2008}.

\section{Minería de datos}

La minería de datos se define como la aplicación de técnicas de inteligencia
artificial sobre grandes cantidades de datos, con el objetivo de descubrir
tendencias, patrones o relaciones ocultas.

Estos descubrimientos suelen ser usados para describir, resumir o clasificar en
grupos un conjunto de datos, además de para predecir en que grupo del conjunto
encajarían nuevos ejemplos de los datos.

Las fases del proceso de minería de datos se pueden dividir en las siguientes:
\begin{enumerate}
	\item \textbf{Selección}: a partir del conjunto de datos original seleccionar
	los ejemplos con los que se va a trabajar.
	\item \textbf{Preprocesamiento}: aplicar operaciones sobre los datos para
	eliminar ruido o medidas erróneas.
	\item \textbf{Transformación}: transformar los datos preprocesados a un formato
	sobre el que poder aplicar las técnicas de minería de datos.
	\item \textbf{Minería de datos}: aplicar algoritmos sobre los datos capaces de
	extraer patrones ocultos en ellos.
	\item \textbf{Evaluación}: comprobar como de bien se ajustan los patrones
	descubiertos a datos nuevos.
\end{enumerate}

Este proyecto se centra en la parte de preprocesamiento, minería de datos y
evaluación.

\subsection{Preprocesamiento}

En esta sección se definen las operaciones de preprocesamiento ofrecidas en la
aplicación. Se van a explicar tomando como ejemplo uno de los espectros usados
por la aplicación.

\subsubsection{Recorte (\textit{Crop})}

La operación de recorte devuelve los valores del espectro contenidos entre un
límite inferior y un límite superior (ver figura~\ref{fig:recorte}).

\imagen{recorte}{Recorte entre valores 200 y 1500}

\subsubsection{Corrección de línea base (\textit{Baseline correction})}

La línea base de un gráfico se puede definir como una línea imaginaria sobre la
que se apoyan los datos. El hecho de que esta línea no sea horizontal puede dar 
lugar a interpretaciones erróneas de los datos, por lo que es conveniente 
eliminarla de los gráficos. Como se ve en la figura~\ref{fig:baseline}, antes 
de aplicar la operación, los datos van ascendiendo a medida que aumenta el eje 
X, después de la operación los datos están correctamente nivelados.

\imagen{baseline}{Corrección de la línea base}

\subsubsection{Normalización (\textit{Normalization})}

La normalización consiste en el proceso de transformar los valores de los
espectros de tal forma que todos los espectros se midan por la misma escala para
poder ser comparados. Como se ve en la figura~\ref{fig:norm} la escala ha
cambiado al rango (0, 1).

\imagen{norm}{Normalización de los datos}

\subsubsection{Compresión (\textit{Squash})}

Esta operación consiste en la construcción de datos similares a los originales
pero de menor tamaño, sin embargo estos datos nuevos deben producir un resultado
casi igual al original al ser analizados. Como se ve en la
figura~\ref{fig:squash}, el gráfico representado parece el mismo pero la escala
es menor, indicando la transformación de los datos.

\imagen{squash}{Compresión de los datos}

\subsubsection{Suavizado (\textit{Smooth})}

La operación de suavizado está enfocada a eliminar el ruido del espectro. Este
ruido provoca picos en el gráfico que impiden analizarlo bien ya que 
pueden modificar la altura de picos que interesan realmente. Esta operación
necesita de un parámetro que indica cuanto se van a reducir los picos (ver
figura~\ref{fig:smooth}).

\imagen{smooth}{Suavizado con ventana de tamaño 25}

\section{Base de datos NoSQL}

Las bases de datos no relacionales, o NoSQL (tradicionalmente ``non SQL'',
actualmente ``Not Only SQL'')~\cite{wiki:nosql} son sistemas de almacenamiento 
de datos que no requieren de una estructura definida y/o fija para su 
funcionamiento.

Se diferencian principalmente de de los sistemas de bases de datos relacionales
en que no necesitan tener definido un esquema al que se ajusten los datos, si no
que estos pueden tener la estructura que necesiten y no necesariamente la misma
a otros datos almacenados.

Los principales tipos en los que se dividen este tipo de sistemas son:
\begin{itemize}
	\item Bases de datos documentales.
	\item Bases de datos clave/valor.
	\item Bases de datos en grafo.
	\item Bases de datos orientadas a objetos.
\end{itemize}

%\tablaSmall{Herramientas y tecnologías utilizadas en cada parte del proyecto}{l
%	c c c c}{herramientasportipodeuso}
%{ \multicolumn{1}{l}{Herramientas} & App AngularJS & API REST & BD & Memoria
%	\\}{ 
%	HTML5 & X & & &\\
%	CSS3 & X & & &\\
%	BOOTSTRAP & X & & &\\
%	JavaScript & X & & &\\
%	AngularJS & X & & &\\
%	Bower & X & & &\\
%	PHP & & X & &\\
%	Karma + Jasmine & X & & &\\
%	Slim framework & & X & &\\
%	Idiorm & & X & &\\
%	Composer & & X & &\\
%	JSON & X & X & &\\
%	PhpStorm & X & X & &\\
%	MySQL & & & X &\\
%	PhpMyAdmin & & & X &\\
%	Git + BitBucket & X & X & X & X\\
%	Mik\TeX{} & & & & X\\
%	\TeX{}Maker & & & & X\\
%	Astah & & & & X\\
%	Balsamiq Mockups & X & & &\\
%	VersionOne & X & X & X & X\\
%} 


	\capitulo{4}{Técnicas y herramientas}

\section{Dash}
Librería en Python que permite crear sitios webs completos para representación de datos. Para ello hace uso de diversas tecnologías, \textit{Flask} para el servidor web, \textit{Plotly} para la representación y \textit{React} para los componentes y actualización.
\subsection{Pros}
\begin{itemize}
	\item Gráficos interactivos
	\item Fácil actualización del gráfico en la web mediante \verb|@app.callback|
	\item Integración de elementos HTML para la actualización del gráfico
	\item Uso de la librería \textit{cufflinks} para unir generar una figura directamente de un \textit{DataFrame}
	\item Al ser de los creadores de \textit{Plotly} y usarlo internamente da la posibilidad de usar sus componentes
	\item Al usar \textit{Flask} como servidor tiene acceso a todas sus ventajas
\end{itemize}
\subsection{Contras}
\begin{itemize}
	\item El código HTML hay que escribirlo desde el código de Python, esto hace que se complique el mantenimiento
	\item No se pueden reutilizar las plantillas de \textit{Flask}
\end{itemize}

\section{Plotly}
Plataforma para representación de datos, dispone de varias librerías para diferentes lenguajes de programación. Representación online y offline.
\subsection{Pros}
\begin{itemize}
	\item Gráficos interactivos
	\item Posibilidad de uso con \textit{Flask} y \textit{Jupyter}
\end{itemize}
\subsection{Contras}
\begin{itemize}
	\item Para representar en la web hay que hacer uso de dos versiones de la librería, para Python y para JavaScript
	\item La representación online guarda los gráficos generados en una cuenta asociada de la plataforma
	\item La representación offline devuelve el gráfico en Python, pero para representarlo es necesario convertirlo a JSON, enviarlo a la web y que la parte de JS lo represente
	\item La actualización es necesaria hacerla desde el cliente con JS, donde no se dispone de los datos ni de las utilidades de minería de datos
\end{itemize}

\section{Jupyter Notebook}
Aplicación web que permite la edición y ejecución de código, Python en este caso, en el navegador, donde también se muestran el resultado de la ejecución. Dispone de \textit{widgets} para interactuar con el programa. Se instala localmente.
\subsection{Pros}
\begin{itemize}
	\item Fácil subir archivos al servidor en el menú principal
	\item Al no tener que hacer una interfaz web permite centrarse en la programación del código de minería de datos
	\item Los gráficos generados con \textit{Plotly} se representan directamente en el notebook
	\item Posibilidad de usar \url{https://mybinder.org/} para el despliegue
	\item Actualización del gráfico por medio de los \textit{widgets} e \verb|interact|
\end{itemize}
\subsection{Contras}
\begin{itemize}
	\item Menos usable e intuitivo
	\item Al estar el código expuesto el cliente podría alterarlo sin querer
	\item \href{http://jupyter-notebook.readthedocs.io/en/latest/public_server.html}{Solo se puede un usuario en servidor público}
\end{itemize}

\section{Flask}
Microframework para aplicaciones web en Python. Aunque por si solo \textit{Flask} no sea muy completo, dispone de una gran cantidad de extensiones oficiales y de la comunidad para suplir todas las características de un framework web completo.
\subsection{Pros}
\begin{itemize}
	\item Maneja bien la subida de ficheros
	\item Al ser web hay más control sobre lo que puede hacer el usuario y sobre lo que se le presenta, con la finalidad de hacer más usable la aplicación
	\item Reutilización de código HTML mediante plantillas y macros
\end{itemize}
\subsection{Contras}
\begin{itemize}
	\item Mucho más trabajo al tener que diseñar y programar la interfaz web
\end{itemize}

	\capitulo{5}{Aspectos relevantes del desarrollo del proyecto}

En este apartado se van a recoger los aspectos más importantes del desarrollo
del proyecto. Desde las decisiones que se tomaron y sus implicaciones,
hasta los numerosos problemas a los que hubo que enfrentarse y cómo se
solucionaron.

\section{Elección del proyecto}

A finales del curso pasado se organizó una charla en la que los profesores iban
a presentar las optativas que daban clase de forma que los alumnos tuviéramos
más fácil elegir asignaturas. En la presentación de la asignatura ``Minería de
Datos'' despertó interés por el tema y se preguntó a José Francisco, por haber
realizado la presentación, sobre TFGs relacionados con el tema.

De los trabajos disponibles este llamó la atención por estar relacionados con
geología y con desarrollo web, además de poder aplicar técnicas minería de datos
en un entorno real de investigación.

\section{Formación}

Para poder realizar el proyecto se necesitaban unos conocimientos no adquiridos
sobre desarrollo web, tanto de la parte de servidor en Flask como la parte del
cliente en HTML, CSS y JavaScript, aunque en menor medida por haberse tocado
algo durante el grado. Como se había hablado con los tutores antes de verano
sobre el proyecto, se dedicó parte a aprender sobre ello. Además de para
aprendizaje, los recursos se han usado también como material de consulta durante
el desarrollo.

\noindent Para la parte del servidor se siguieron los libros y tutoriales:
\begin{itemize}
	\item Flask Web Development\cite{grinberg2014flask}
	\item Explore Flask\cite{exploreflask}
	\item The Flask Mega-Tutorial Legacy (2012)\cite{grinberg-mega-legacy}
	\item The Flask Mega-Tutorial (2017)\cite{grinberg-mega}
\end{itemize}

\noindent Para la parte del cliente se utilizaron principalmente los siguientes
materiales:
\begin{itemize}
	\item MDN Web Docs\cite{mdn}
	\item W3Schools Tutorials\cite{w3schools}
\end{itemize}

A medida que se añadían nuevas herramientas al proyecto, su documentación
oficial también ha sido consultada en varias ocasiones, están disponibles en:
\begin{itemize}
	\item Documentación de Flask\cite{doc:flask}
	\item Documentación de Bootstrap\cite{doc:bootstrap}
	\item Documentación de Nanobox\cite{doc:nanobox}
	\item Documentación de PyMongo\cite{doc:pymongo}
	\item Documentación de Dash\cite{doc:dash}
	\item Documentación de Plotly\cite{doc:plotly}
\end{itemize}

\section{Sistema de usuarios}

Uno de los primeros problemas que se plantearon fue la forma de ofrecer el
sistema de usuarios, para que cada uno pudiera almacenar sus archivos. Las
opciones que se presentaban eran implementar uno desde cero con ayuda de las
extensiones que ofrece Flask o mediante un sistema de terceros, como puede ser
Google.

Implementar el sistema desde cero tenía la ventaja de que los usuarios no tenían
que salir de la página para iniciar sesión y que se había aprendido como hacerlo
en los tutoriales sobre Flask mencionados anteriormente, mientras que el sistema
de terceros no se sabía como hacerlo, pero tenía más ventajas, siendo las
principales no tener que mantener una base de datos de usuarios, no depender de
un sistema de envío de correo electrónico (dio problemas en proyectos
anteriores) y no tener que obligar a los usuarios a crearse otra cuenta al poder
usar una existente.

Al final se decidió usar la autenticación con Google, por estar casi garantizado
que los usuarios van a tener una cuenta existente, la documentación sobre este
aspecto es abundante y era fácil de implementar. Además al iniciar sesión con
este servicio nos permite usar la API de Google para obtener los datos del
usuario necesarios.

\section{Subida de ficheros}\label{sec:subida}

Una de las primeras características que se implementó fue la subida de ficheros
al servidor, al tener ya el sistema de usuarios se podían organizar los datos
sin problema para cada cliente. El formato requerido para subir los datos pasó
por varios cambios motivados por la forma en la que Susana nos enviaba los datos
que probar la aplicación. En secciones posteriores se explica como estos cambios
afectaron también al almacenamiento de los datos y la interfaz del visor.

\subsection{Organización en directorios}\label{sec:directorios}

Al principio los datos se encontraban organizados en directorios cuyo nombre
proporcionaba toda la información sobre los ejemplos que contenían. Por lo que
el formato que se pedía era simple, un archivo comprimido en formato ``.zip''
que contuviera estos directorios.

\subsection{Estructura según hoja de metadatos}\label{sec:excel}

Sobre mediados de Abril los nuevos datos recibidos empezaron a estar organizados
según una hoja de Excel que relacionaba los nombres de directorios y los
ejemplos que contenían con sus metadatos, en vez de estar contenidos en el
nombre.

Esta hoja contenía una columna llamada ``Id'' con el nombre de un directorio y
los ejemplos de ese tipo, las siguientes columnas contenían datos sobre los
ejemplos contenidos en esa fila, como la mina de la que han sido extraídas las
muestras o la profundidad.

Esta hoja se adaptó para aplicación manteniendo la columna ``Id'' y añadiendo
fijas una columna para la mina y otras dos para la profundidad, una nominal y
otra numérica. El formato de subida pasó a ser en un mismo fichero ``.zip'' los
directorios que contienen los ejemplos y la hoja con los metadatos previamente
cumplimentada.

\section{Almacenamiento de datos}

El principal motivo de implementar un sistema de usuarios fue que cada uno
pudiera tener sus datos almacenados para no tener que subirlos cada vez que se
quiera trabajar con ellos.

\subsection{Inicialmente}

Al principio se almacenaban directamente en un directorio nombrado como el id
del usuario y lo único que se hacía era descomprimir el archivo comprimido en
ese directorio. Esta forma de almacenar los datos conlleva varios problemas que
motivaron el cambio en la forma de almacenamiento a la actual. Esta forma de
almacenamiento se corresponde al formato explicado en la
sección~\ref{sec:directorios} (Organización en directorios).

El problema más evidente, y que más notarían los usuarios, es que cada vez que
se quisiera ver un espectro hay que cargarlo desde el disco, disminuyendo el
rendimiento general de la aplicación.

Otro gran problema de esta forma de almacenar era como manejaba el servidor los
ficheros guardados, este tema se explicará más en detalle en la
sección~\ref{sec:despliegue}.

El último problema y el más problemático desde el punto de vista de programación
es que para cada operación que se quisiera realizar con los datos se necesitaban
escribir funciones auxiliares complejas para buscar y manipular el árbol de
directorios, las cuáles había que modificar con cada cambio en la estructura,
dificultando el mantenimiento de la aplicación.

Se planteó cambiar la forma de almacenar los datos en el servidor, cambio que
terminó por realizarse después de valorar ventajas y desventajas en este momento
del desarrollo.

\subsection{Migración a MongoDB}

Por sugerencia del tutor se investigó la posibilidad de usar MongoDB para
almacenar los datos. Se vio que podría funcionar realmente bien al poder
insertar directamente los ficheros cargados en la base de datos, recuperarlos y
borrarlos fácilmente. Como consecuencia directa del cambio los métodos
auxiliares con los que se interactuaba con los datos disminuyeron
considerablemente en tamaño y complejidad.

En primera instancia se optó por guardar los conjuntos de datos como un
documento en el que estaban contenidos varios \textit{DataFrames} (cada uno
representando un espectro) agrupados según la carpeta en la que estuvieran
localizados los espectros. Cada \textit{DataFrame} contenía dos columnas, una
con los valores del eje X y otra con los valores del eje Y.

Pero seguido de adoptar esta estructura y tener la aplicación adaptada para ello
llegó el cambio en como recibíamos los datos, provocando otra modificación en la
estructura. Esta vez se decidió agrupar todo el conjunto de datos subido en un
solo \textit{DataFrame}, en el que cada fila se corresponde con un espectro, las
columnas representan el eje X y los valores de cada fila en esas columnas
representan el valor del eje Y. Cada fila contiene adicionalmente columnas que
indican el nombre del espectro y sus metadatos asociados (ver
Figura~\ref{fig:dataframe-def}).

\begin{figure}[!h]
	\centering
	\includegraphics[width=\textwidth]{dataframe-def}
	\caption{Estructura definitiva}\label{fig:dataframe-def}
\end{figure}

Guardar los datos de esta forma facilita aplicar los métodos de minería de datos
al estar pensado para aplicarse en lote.

\section{Interfaz del visor}

Esta parte del proyecto está muy relacionado con lo explicado en la
sección~\ref{sec:subida}, el usuario del visor espera que se le presenten los
datos de una forma similar a como los ha subido.

Cuando los datos se subían como un archivo comprimido con carpetas cuyo nombre
contiene la información se presentaban al usuario dos menús desplegables. En la
carga solo uno de ellos contenía valores, los nombres de los directorios. El
segundo desplegable actualizaba sus valores con los nombres de los ficheros
contenidos en la carpeta seleccionada, al seleccionar el fichero se cargaba en
el visor.

Con el cambio de la hoja con metadatos, el tutor José Francisco propuso que al
usuario se le mostrase una tabla parecida al hoja que había rellenado antes de
subir los datos, con la diferencia de que se mostrasen todos los ficheros como
filas con sus metadatos, asemejándose a la estructura final de almacenamiento de
datos.

Mostrar los datos al usuario en forma de tabla trajo las ventajas de poder
representar varios espectros a la vez, filtrar y ordenar según las columnas (ver
Figura~\ref{fig:tabla-espectros}).

\imagen{tabla-espectros}{Tabla de espectros} 

\section{Despliegue}\label{sec:despliegue}

La idea de realizar el proyecto como una aplicación web tiene relación más que directa con que pueda estar accesible en pocos clicks. Para ello se necesita que esté desplegada y accesible en Internet. En esta sección de describe las etapas por las que pasó el despliegue, los problemas que surgieron y como se solucionaron.

En las primeras reuniones del proyecto se comentó con los tutores que un gran problema de los proyectos anteriores desarrollados en web se centraban en el despliegue al final del proyecto, haciendo que alguna vez no pudieran llegarse a desplegar, por eso se planteó la idea de empezar a desplegar desde el inicio del proyecto.

\subsection{Servidor}

La plataforma escogida fue \href{https://www.heroku.com/}{Heroku}\footnote{\url{https://www.heroku.com/}} por su simplicidad y un plan gratuito que cubre las necesidades del proyecto. Aunque en primera instancia parecía que esta plataforma funcionaba bien para nuestras necesidades se vio que no contaba con almacenamiento persistencia, convirtiendo su uso en inviable.

Las opciones que se plantearon fueron buscar una forma de añadir este almacenamiento y cambiar de servidor por completo. Para añadirlo en Heroku había que depender de \textit{plugins} de terceros para enlazar servicios de almacenamiento de terceros teniendo que modificar la aplicación para hacerlo funcionar, además de ser servicios de pago.

Al final se escogió por cambiar a un proveedor \textit{cloud} de pago que ofreciese máquinas virtuales privadas, las opciones manejadas fueron \href{https://www.digitalocean.com/}{DigitalOcean}\footnote{\url{https://www.digitalocean.com/}}, \href{https://aws.amazon.com/es/}{Amazon Web Services}\footnote{\url{https://aws.amazon.com/es/}} y \href{https://cloud.google.com/}{Google Cloud}\footnote{\url{https://cloud.google.com/}}. Se escogió la primera opción ya que gracias al \href{https://education.github.com/pack}{Student Developer Pack}\footnote{\url{https://education.github.com/pack}} se disponía de un cupón de 50\$ en esta plataforma.

\subsection{Herramientas para el despliegue}

La plataforma Heroku posee sus propias herramientas para el despliegue, facilitando en gran cantidad este proceso. Esta plataforma un repositorio \textit{git} remoto para almacenar la aplicación por lo que al contar ya con este sistema de control de versiones en el proyecto no hubo que modificar casi nada para poder desplegar, pero por lo problemas comentados anteriormente se tuvo que abandonar.

Para complementar a DigitalOcean y facilitar la tarea del despliegue se ha usado el servicio \href{https://nanobox.io/}{Nanobox}\footnote{\url{https://nanobox.io/}}. Al principio de usar esta plataforma se vio que los datos almacenados se eliminaban en cada despliegue, para ello hubo que añadir un segundo contenedor que se ocupara del almacenamiento, lo bueno es que este contenedor se asocia a un directorio del servidor, por lo que la aplicación no se tuvo que modificar. Más adelante se añadió otro contenedor para gestionar la base de datos MongoDB.

Las mayores ventajas que posee esta plataforma es el despliegue con un solo comando, se encarga de toda la configuración, monitorización del servidor, conexión remota al servidor sencilla y escalado sencillo del servidor.

\section{Procesamiento}

\section{Cohesión entre Flask y Dash}




	\capitulo{6}{Trabajos relacionados}

Este apartado sería parecido a un estado del arte de una tesis o tesina. En un
trabajo final grado no parece obligada su presencia, aunque se puede dejar a
juicio del tutor el incluir un pequeño resumen comentado de los trabajos y
proyectos ya realizados en el campo del proyecto en curso.

En este apartado se hace un pequeño resumen o mención de otras herramientas o
artículos sobre el tema que puedan estar relacionados con el tema tratado en el
proyecto.

\section{Librerías}
\subsection{Scikit-spectra}
Al principio del proyecto se habló mucho con los tutores de probar y usar esta
librería como referencia o como base del proyecto. Pero después de analizar su
repositorio se vio que llevaba tiempo sin mantenimiento y al instalar y probar
los ejemplos que trae incluidos salían errores por todas partes, por lo que se
dejó de lado. La principal característica de la librería es la visualización y
la construcción de interfaces gráficas mediante \textit{IPython Notebooks} para
su ejecución en navegador\cite{art:skspec}.

	\capitulo{7}{Conclusiones y Líneas de trabajo futuras}

En esta sección se exponen las conclusiones obtenidas al terminar el desarrollo del proyecto y se comentan anotaciones que podrían seguirse para avanzar con este trabajo en el futuro.

\section{Conclusiones}

\section{Líneas de trabajo futuras}

Este trabajo esta pensado para evolucionar en el futuro hacia un proyecto más ambicioso de tal forma que sea posible su uso en otros ámbitos aparte de la geología. A continuación se presenta una lista de tareas a realizar para continuar con su desarrollo:
\begin{itemize}
	 \item Modificar la aplicación de forma que sea capaz de trabajar con atributos definidos por el usuario, en vez de que estos estén fijos.
	 \item Mejorar la creación de clasificadores para poder elegir sobre que atributo crearlos, en vez de todos a la vez.
	 \item Poder aplicar un preprocesamiento de los datos antes de crear el clasificador en vez de usar uno fijo y por defecto.
	 \item Crear unos tests más exhaustivos.
	 \item Mejorar el sistema de usuarios añadiendo opciones como ``Eliminar cuenta'' o similares.
	 \item Poder poner tareas que requieran de más tiempo en segundo plano para que el usuario pueda seguir usando la aplicación.
	 \item Teniendo en cuenta el punto anterior, añadir un sistema de notificaciones por correo electrónico para avisar al usuario que las tareas que estaban en segundo plano han terminado.
\end{itemize}


	
	
	\bibliographystyle{plain}
	\bibliography{bibliografia}
	
\end{document}


