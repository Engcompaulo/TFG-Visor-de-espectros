\capitulo{7}{Conclusiones y Líneas de trabajo futuras}

En esta sección se exponen las conclusiones obtenidas al terminar el desarrollo del proyecto y se comentan anotaciones que podrían seguirse para avanzar con este trabajo en el futuro.

\section{Conclusiones}

\section{Líneas de trabajo futuras}

Este trabajo esta pensado para evolucionar en el futuro hacia un proyecto más ambicioso de tal forma que sea posible su uso en otros ámbitos aparte de la geología. A continuación se presenta una lista de tareas a realizar para continuar con su desarrollo:
\begin{itemize}
	 \item Modificar la aplicación de forma que sea capaz de trabajar con atributos definidos por el usuario, en vez de que estos estén fijos.
	 \item Mejorar la creación de clasificadores para poder elegir sobre que atributo crearlos, en vez de todos a la vez.
	 \item Poder aplicar un preprocesamiento de los datos antes de crear el clasificador en vez de usar uno fijo y por defecto.
	 \item Crear unos tests más exhaustivos.
	 \item Mejorar el sistema de usuarios añadiendo opciones como ``Eliminar cuenta'' o similares.
	 \item Poder poner tareas que requieran de más tiempo en segundo plano para que el usuario pueda seguir usando la aplicación.
	 \item Teniendo en cuenta el punto anterior, añadir un sistema de notificaciones por correo electrónico para avisar al usuario que las tareas que estaban en segundo plano han terminado.
\end{itemize}

