\lstset{
	frame=single
}

\apendice{Documentación del programador}

\section{Introducción}
En este apéndice se presenta todo lo que tiene que conocer un desarrollador 
para poder continuar con el desarrollo de la aplicación. Se describe la 
estructura de directorios del proyecto, como instalar la aplicación

\section{Estructura de directorios}

A continuación se presentan los directorios que contiene el proyecto con una 
breve descripción de cada uno.\\

\dirtree{%
.1 / Directorio raíz.
.2 Documentacion LaTeX/ -Documentación del proyecto.
.3 img/ -Imágenes de la documentación.
.3 tex/ -Secciones de la documentación.
.3 anexos.pdf -Anexos del proyecto.
.3 memoria.pdf -Memoria del proyecto.
.2 etc/ -Configuración del servidor.
.2 SpectraViewer/ -Aplicación web.
.3 auth/ -Módulo de autenticación.
.4 \_\_init\_\_.py -Fichero principal del módulo.
.4 routes.py -Rutas del módulo.
.3 main/ -Módulo principal de la aplicación.
.4 \_\_init\_\_.py -Fichero principal del módulo.
.4 errors.py -Manejadores de errores de la aplicación.
.4 forms.py -Formularios de WTForms.
.4 routes.py -Rutas del módulo.
.3 processing/ -Fuentes de la librería superman ampliada.
.3 static/ -Contenido estático de la aplicación.
.4 css/ -Directorio con hojas de estilos.
.4 img/ -Imágenes usadas en la web.
.4 js/ -Directorio con ficheros de JavaScript.
.3 templates/ -Plantillas de Jinja2.
.4 errors/ -Plantillas específicas de errores.
.3 utils/ -Directorio con utilidades de la aplicación.
.3 visualization/ -Aplicaciones Dash para la visualización.
.4 \_\_init\_\_.py -Fichero principal del módulo.
.4 common.py -Cosas comunes en la visualización.
.4 dataset.py -Visualización de datasets.
.4 spectrum.py -Visualización de espectros.
.3 \_\_init\_\_.py -Fichero principal del módulo.
.3 manual\_de\_uso.pdf -Manual de usuario.
.3 metadatos.xlsx -Plantilla de metadatos.
.3 models.py -Modelos de la aplicación.
.2 tests/ -Directorio de las pruebas.
.2 boxfile.yml -Configuración de Nanobox.
.2 config.py -Configuración de la aplicación web.
.2 install\_dependencies -Script para instalar dependencias.
.2 requirements.txt -Fichero con las dependencias del proyecto.
.2 start.py -Script para ejecutar la aplicación.
}

\section{Manual del programador}

En esta sección se explican los puntos más importantes que los desarrolladores 
deben tener en cuenta para mantener o seguir ampliando este proyecto.

\subsection{Definir funciones de actualización}

Dentro de la parte de visualización, las funciones de actualización son una 
parte esencial, por lo que un punto importante es saber codificarlas 
correctamente.

Para empezar, son funciones de Python asociadas a elementos HTML de la interfaz 
que se ejecutan cuando una propiedad seleccionada del elemento se modifica. 
Esta asociación se realizado por medio de un decorador, que recibe como 
parámetros un elemento que actualizar y una lista de elementos que tomar como 
entradas del método. El siguiente método se encarga de mostrar los espectros 
cuando se seleccionan en la tabla.
\begin{lstlisting}[caption=Carga de espectros al ser seleccionados,frame=single,
breaklines=true,
language=Python,
tabsize=2,
captionpos=b,
keywordstyle=\color{blue},
stringstyle=\color{red}]
import dash
import dash_core_components as dcc
import dash_html_components as html
import dash_table_experiments as dt
from dash.dependencies import Input, Output

@app.callback(Output('spectrum-original', 'figure'),
			[Input('metadata', 'rows'),
			Input('metadata', 'selected_row_indices')])
def update_spectrum(rows, spectra_index):
	dataset = session['current_dataset']
	user_id = session['user_id']
	dataset_data = get_user_dataset(dataset, user_id)
	figure = {
		'layout': {
			'title': 'Espectro original',
			'xaxis': {'title': 'Raman shift'},
			'yaxis': {'title': 'Intensity'}
		},
		'data': list()
	}
	for i in spectra_index:
		name = rows[i]['Nombre']
		spectrum = dataset_data[dataset_data['Nombre'] == name]
		spectrum = spectrum.drop(
			columns=['Nombre', 'Etiqueta', 'Mina', 'Profundidad',
				'Profundidad_num'])
		figure['data'].append({
			'x': spectrum.columns.tolist(),
			'y': spectrum.values[0],
			'name': f'{name}'
		})
	return figure
\end{lstlisting}

En el decorador se recibe una elemento \code{Output}, con parámetros 
\code{spectrum-original}, como elemento de salida, el grafico con el espectro 
original, y \code{figure} como la propiedad a modificar del elemento.

Como entradas se reciben dos \code{Input}, el primero corresponde a la tabla 
con los espectros, con la propiedad de las filas, y el segundo se refiere 
también a la tabla, pero con la propiedad asociada a las filas seleccionadas. 
Cada vez que se modifique una de esas propiedades se ejecutaría el método. Los 
parámetros que recibe el método son los valores de las propiedades definidas en 
los \code{Input}

Al terminar la ejecución del método, el valor devuelto ocupará el valor de la 
propiedad definida en el \code{Output}, en este caso la propiedad \code{figure} 
del espectro original.

\subsection{Manual de despliegue}

La configuración de despliegue tiene tres etapas, que se describen en esta 
sección.

\subsubsection{Alojamiento}

Para este proyecto el proveedor escogido ha sido 
\hrefFootnote{https://www.digitalocean.com/}{Digital Ocean}, pero puede servir 
cualquiera de los 
\hrefFootnote{https://docs.nanobox.io/providers/hosting-accounts/}{proveerdores 
compatibles} con Nanobox.

Nos dirigimos a la página del proveedor para registrarnos y poder conseguir un 
\textit{token} que Nanobox nos pedirá más adelante. Es necesario asociar una 
tarjeta al servicio para que nos puedan cobrar el coste del servidor.

El \textit{token} lo creamos en la sección API de la barra de navegación.

\subsubsection{Nanobox}

A continuación, crearemos una cuenta en el servicio \hrefFootnote{ 
https://nanobox.io/}{Nanobox}. Con la cuenta ya confirmada, nos dirigimos a las 
opciones de la cuenta y asociamos la cuenta de Digital Ocean con el 
\textit{token} obtenido previamente.

El siguiente paso es instalar la herramienta 
\hrefFootnote{https://docs.nanobox.io/install/}{nanobox}. La primera que la 
usemos nos pedirá los datos de nuestra cuenta para iniciar sesión.

La configuración de este servicio se realiza por medio del fichero 
\code{boxfile.yml}, aquí se indican los componentes de la aplicación y su 
configuración.

Para poder desplegar la aplicación, necesitamos tener creado una aplicación en 
Nanobox, esto se realiza desde la web. Para ello, pulsamos el botón ``Launch 
New App'' y seguimos las instrucciones que se muestran.

Ahora estando todo configurado queda desplegar la aplicación, este paso se 
realiza con el siguiente comando:
\begin{lstlisting}
	$ nanobox deploy
\end{lstlisting}

Para cualquier otra duda con la herramienta se puede buscar en su documentación 
oficial~\cite{doc:nanobox}.

\section{Instalación y ejecución del proyecto}\label{sec:instalacion}

Dado que el proyecto se ha desarrollado usando el sistema operativo Ubuntu, las 
instrucciones de instalación están orientadas a ese sistema.

\subsection{MongoDB}

Antes de instalar el proyecto es necesario instalar la base de datos 
\hrefFootnote{https://www.mongodb.com/}{MongoDB}.

Una vez instalado, para iniciar el servicio y poder conectarnos a la base de 
datos es necesario ejecutar el siguiente comando:
\begin{lstlisting}
	$ sudo service mongod start
\end{lstlisting}

\subsection{Python}

Este proyecto está desarrollado con la versión 3.6.3, pero con para desarrollar 
la mínima es 3.6. Python se puede descargar desde el siguiente enlace: 
\url{https://www.python.org/downloads/}

\subsection{Instalación}

La forma más cómoda de obtener el código del proyecto es mediante \textit{git}, 
para ello usar el siguiente comando:
\begin{lstlisting}
	$ git clone <url_del_repositorio>
\end{lstlisting}
Siendo \url{https://github.com/IvanBeke/TFG-Visor-de-espectros.git} la URL del 
repositorio.

Ya con el proyecto descargado, el siguiente paso es instalar las dependencias. 
Aunque se pueden instalar sobre la instalación global de Python, es 
recomendable usar un entorno virtual. Para crearlo usar el siguiente comando:
\begin{lstlisting}
	$ python3 -m venv <nombre_del_entorno>
\end{lstlisting}
Para activar el entorno usar el siguiente comando:
\begin{lstlisting}
	$ source <ruta/del/entorno>/bin/activate
\end{lstlisting}
Para instalar las dependencias se pueden ejecutar cualquiera de los siguientes 
comandos:
\begin{lstlisting}
	$ bash install_dependencies.sh
	$ pip install -r requirementes.txt
\end{lstlisting}

\subsubsection{Claves OAuth de Google}

El inicio de sesión con Google usa la API de OAuth, para que funcione se 
necesitan el \textit{client\_id} y \textit{client\_secret}. Estos valores se 
obtienen desde \url{https://console.developers.google.com/apis/credentials}.

\subsubsection{Variables de entorno}

Para que la aplicación funcione correctamente hay que definir las siguientes 
variables de entorno:
\begin{itemize}
	\item \code{GOOGLE\_OAUTH\_CLIENT\_ID}: \textit{client\_id} obtenido 
	anteriormente.
	\item \code{GOOGLE\_OAUTH\_CLIENT\_SECRET}: \textit{client\_secret} 
	obtenido anteriormente.
	\item \code{UPLOAD\_FOLDER}: directorio donde se van a guardar los archivos 
	que se suban a la aplicación.
	\item \code{SECRET\_KEY}: clave necesaria por Flask.
	\item \code{OAUTHLIB\_RELAX\_TOKEN\_SCOPE}: recomendable poner este valor a 
	1\footnote{\url{https://flask-dance.readthedocs.io/en/v0.14.0/quickstarts/google.html}}.
	\item \code{ENVIRONMENT}: entorno en el que se está actualmente, 
	development, production.
	\item \code{DATA\_MONGODB\_HOST}: dirección para conectarse a MongoDB.
\end{itemize}

\subsection{Ejecución}

Para ejecutar la aplicación es necesario tener el entorno virtual activo y 
ejecutar el siguiente comando:
\begin{lstlisting}
	$ python start.py
\end{lstlisting}
Esto lanza el servidor Flask en el ordenador local y puerto 5000, se accede 
desde \url{https://127.0.0.1:5000/}. Al entrar seguramente el navegador avise 
de certificado incorrecto, esto se debe al uso de un certificado local, ya que 
OAuth requiere el uso de \code{https}.

\section{Pruebas del sistema}

Las pruebas del sistema se encuentran dentro del directorio \code{tests}. Se 
han codificado pruebas unitarias para probar los modelos de la aplicación, se 
encuentran en el fichero \code{unitarios}. Se ha usado el \textit{framework} 
\textit{Unittest}~\cite{unittest}, que viene por defecto instalado con Python, 
por lo que no es necesario instalar ninguna dependencia.

Para ejecutar las pruebas, dirigirse al directorio raíz del proyecto y ejecutar 
el siguiente comando:
\begin{lstlisting}
	$ python -m unittest tests.unitarios
\end{lstlisting}