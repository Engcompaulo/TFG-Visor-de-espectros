\apendice{Documentación de usuario}

\section{Introducción}

En este apéndice se explica los requisitos que debe cumplir el usuario para 
ejecutar la aplicación, como lanzarla y como usarla.

\section{Requisitos de usuarios}

Al tratarse de una aplicación los requisitos que debe cumplir el usuario son 
los siguientes:
\begin{itemize}
	\tightlist
	\item Navegador web instalado.
	\item Cuenta de Google activa.
	\item JavaScript avtivo en el navegador.
	\item \textit{Cookies} activas en el navegador.
\end{itemize}

La aplicación se ha probado en los siguientes navegadores y se certifica que 
funciona:
\begin{itemize}
	\tightlist
	\item Google Chrome 67.0.3396.99.
	\item Chromium 66.0.3359.181.
	\item Mozilla Firefox 60.0.2.
\end{itemize}

Aunque la aplicación funciona en dispositivos móvil no se recomienda hacerlo 
debido a que la visualización sería demasiado pequeña y no estarían disponibles 
las acciones sobre los gráficos.

\section{Instalación}

Debido a que se proporciona una aplicación web no es necesario instalarla para 
poder usarla. Sin embargo, si se quiere proceder a la instalación se pueden 
seguir las instrucciones en la sección~\ref{sec:instalacion}.

De cara a probar la aplicación con ejemplos ya cargados se proporciona un 
cuenta con ejemplos cargados para su uso. El usuario proporcionado es  
\code{tfg.visor.ejemplos@gmail.com} y su contraseña es 
\code{tfg\_visor\_ejemplos}.

Hay que tener en cuenta que al ser una cuenta de prueba la pueden usar varias 
personas por lo que se recomienda encarecidamente no borrar los espectros ni 
los datasets que se encuentran subidos.

\section{Manual del usuario}

En esta sección se enseña al usuario como manejar la aplicación.

\subsection{Inicio}

Nada más entrar a la aplicación se muestra la bienvenida. Si no se ha iniciado 
sesión aparece el mensaje de la imagen~\ref{fig:manual-uso/bienvenida}.

\imagen{manual-uso/bienvenida}{Bienvenida}

Después de iniciar sesión (botón en la esquina superior derecha), o si no se 
había cerrado sesión anteriormente, se muestra el mensaje de la 
imagen~\ref{fig:manual-uso/bienvenida-inicio}.

\imagen{manual-uso/bienvenida-inicio}{Bienvenida después de iniciar sesión}

Las opciones presentadas en la barra de navegación y en las tarjetas situadas 
debajo del mensaje son las mismas, con la excepción de que en las tarjetas 
aparece una pequeña descripción de la acción.

\subsection{Mis ficheros}

Pulsando en ``Mis ficheros'' nos aparece la página con todo lo que tenemos 
asociado en nuestra cuenta (ver figura~\ref{fig:manual-uso/ficheros}), 
organizado en tres columnas, datasets, espectros y clasificadores. Para cada 
entidad se muestra el nombre, los comentarios y las opciones disponibles.

\imagen{manual-uso/ficheros}{Mis ficheros}

\subsection{Subir dataset}

Pulsando en ``Subir dataset'' nos dirigimos a la página en la que podemos subir 
en conjunto de espectros. A la izquierda se muestran las instrucciones a seguir 
y unas notas respecto al formulario, presente a la derecha. En la 
imagen~\ref{fig:manual-uso/subir-dataset} se puede ver la página con el 
formulario completo.

\imagen{manual-uso/subir-dataset}{Página para subir un dataset}

 Después de pulsar el botón ``Subir'' se muestra un mensaje 
de espera mientra se procesa la petición (ver 
figura~\ref{fig:manual-uso/espera}).

\imagen{manual-uso/espera-dataset}{Mensaje de espera al subir un dataset}

Cuando la subida del fichero termine, mostrará la página con los ficheros 
guardados, indicando que el dataset se ha subido correctamente.

\subsection{Subir espectro}

Pulsando en ``Subir espectro'' nos dirigimos a la página en la que podemos 
subir un espectro. A la izquierda se muestran las indicaciones respecto al 
formato requerido en el fichero del espectros. En la 
imagen~\ref{fig:manual-uso/subir-espectro} se puede ver la página con el 
formulario completo.

\imagen{manual-uso/subir-espectro}{Página para subir un espectro}

Después de pulsar el botón ``Subir'' se muestra un mensaje 
de espera mientra se procesa la petición (ver 
figura~\ref{fig:manual-uso/espera}). Sin embargo, como esta operación suele 
tardar poco tiempo, el mensaje no llega a leerse.

\imagen{manual-uso/espera-espectro}{Mensaje de espera al subir un espectro}

Cuando la subida del fichero termine, mostrará la página con los ficheros 
guardados, indicando que el espectro se ha subido correctamente.

\subsection{Eliminado}

Cuando se quiera eliminar cualquiera de los elementos guardados, es suficiente 
con presionar el botón ``Eliminar''. A continuación se muestra una ventana para 
confirmar la acción. Si se acepta la eliminación, se borra el elemento de la 
cuenta y se muestra un mensaje indicando que la acción se ha completado.

\subsection{Visualización y procesamiento}

Para visualizar un dataset o espectro hay que pulsar en el botón ``Visualizar'' 
en el elemento que se desee. Esta acción nos redirige a la vista de 
visualización.

\subsubsection{Datasets}

La visualización de datasets se compone de cuatro elementos 
(figura~\ref{fig:manual-uso/vis-dataset}), la tabla con los espectros 
contenidos, los controles de procesamiento, la visualización del 
espectro original y la visualización del espectro procesado. Se proporciona una 
ayuda integrada en esta vista.

\imagen{manual-uso/vis-dataset}{Visualización de dataset}

Para visualizar un espectro hay que seleccionarlo en la tabla. Pasados unos 
segundos, el espectro original y procesado aparecen en sus gráficos 
correspondientes. La aplicación soporta la visualización de varios espectros a 
la vez para poder compararlos (ver 
figura~\ref{fig:manual-uso/dataset-visualizado}).

\imagen{manual-uso/dataset-visualizado}{Visualización de dataset con espectros}

El espectro procesado se actualiza automáticamente al modificar las opciones de 
procesamiento.

\subsubsection{Espectro}

La visualización de espectro se compone de tres elementos (ver figura~\ref{f})
(figura~\ref{fig:manual-uso/vis-espectro}), los controles de procesamiento, la 
visualización del espectro original y la visualización del espectro procesado. 
Se proporciona una ayuda integrada en esta vista.

\imagen{manual-uso/vis-espectro}{Visualización de espectro}

El espectro procesado se actualiza automáticamente al modificar las opciones de 
procesamiento.

\subsection{Creación de modelos}

Para crear un modelo primero hay que escoger que dataset se quiere usar como 
referencia. Una vez decidido, se pulsa en el botón ``Crear modelo'', esto 
redirige a la página de creación de modelos (ver 
figura~\ref{fig:manual-uso/seleccionar-modelo}). 

\imagen{manual-uso/seleccionar-modelo}{Página de creación de modelos}

Para poder crear el modelo hay que seleccionar uno de los disponibles en el 
desplegable, esta acción hace visible un formulario con los posibles parámetros 
del modelo (ver figura~\ref{fig:manual-uso/modelo-seleccionado} El 
formulario se actualiza cada vez que se cambia el modelo seleccionado.

\imagen{manual-uso/modelo-seleccionado}{Formulario de creación de 
modelos}

Cuando se haya terminado de rellenar el formulario, todos los campos son 
opcionales, hay que pulsar el botón ``Crear y evaluar modelo''. Esto crea el 
modelo con los parámetros introducidos, lo entrena con el dataset seleccionado 
y lo evalúa. Mientras se realiza este proceso, se le muestra un mensaje de 
espera al usuario hasta que se completa la acción (ver 
figura~\ref{fig:manual-uso/espera-modelo}).

\imagen{manual-uso/espera-modelo}{Mensaje de espera en la creación de modelos}

Cuando se termina el entrenamiento, se muestra una página con los resultados de 
la evaluación y un formulario en caso de querer guardar el clasificador (ver 
figura~\ref{fig:manual-uso/resultados}). Ambos botones, ``Guardar'' y 
``Descartar'', llevan de vuelta a la página con los ficheros guardados, pero al 
pulsar el de guardar, guarda el clasificador en la cuenta del usuario mientras 
que pulsar el botón de descarte no.

\imagen{manual-uso/resultados}{Resultados de la evaluación}

