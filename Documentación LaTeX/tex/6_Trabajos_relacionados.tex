\capitulo{6}{Trabajos relacionados}

Como se comentó en la introducción (página~\pageref{ch:introduccion}), el uso de
técnicas de minería de datos está empezando a crecer en este campo. Primero, se
comenta un artículo que habla en profundidad, sobre este tema para a
continuación hablar de herramientas existentes para el análisis de espectros
Raman.

\section{Artículos científicos}

\subsection{Machine learning tools for mineral recognition	and classification
	from Raman spectroscopy~\cite{art:raman-carey}}\label{sec:carey}
En este estudio se prueban a usar técnicas de aprendizaje automático con el
objetivo de mejorar la identificación de materiales usando todo el rango del
espectro. Se enumeran varias técnicas preprocesamiento y algoritmos de
clasificación usados para comprobar con que combinaciones se obtienen mejores
resultados. Las técnicas de procesamiento de este proyecto están basadas en las
indicadas en este artículo.

\section{Software}

\subsection{CrystalSleuth}
Este software desarrollado dentro del proyecto 
\href{http://rruff.info/}{RRUFF}\footnote{\url{http://rruff.info/}}, dedicado a
recopilar en una base de datos espectros Raman, difracción de rayos X y datos
químicos de minerales, permite cargar espectros para su análisis y manipulación,
adicionalmente pueden compararse los espectros cargados con los almacenados para
intentar averiguar a qué pertenecen.

\section{Librerías}

\subsection{Scikit-spectra}
Al principio del proyecto se habló mucho con los tutores sobre probar y usar
esta librería como referencia o como base del proyecto. Pero, después de
analizar su repositorio, se vio que llevaba tiempo sin mantenimiento y al
instalar y probar los ejemplos que trae incluidos salían errores por todas
partes, por lo que se dejó de lado. La principal característica de la librería
es la visualización y la construcción de interfaces gráficas mediante
\textit{IPython Notebooks} para su ejecución en navegador~\cite{art:skspec}.

\subsection{Superman}\label{lib:superman}

Abreviatura de ``SpectrUm PrEpRocessing MAchiNe''. Esta librería alojada en
\hrefFootnote{https://github.com/all-umass/superman}{GitHub} fue usada en el
desarrollo del proyecto publicado en el artículo mencionado anteriormente
(\ref{sec:carey})~\cite{art:raman-carey}, siendo el autor del artículo uno de
los desarrolladores. Se ha usado como librería base para las opciones de
procesamiento, añadiendo funciones según el formato existente.

