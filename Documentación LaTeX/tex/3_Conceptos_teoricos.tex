\capitulo{3}{Conceptos teóricos}

\section{Espectroscopia Raman}\label{def:raman}
La espectroscopia Raman hace uso del fenómeno conocido como dispersión
inelástica de fotones para obtener gráficas que definen la estructura y
composición de un material o elemento.

Este fenómeno se refiere a como los fotones rebotan, o mejor dicho, son
absorbidos y vueltos a emitir. Los fotones se pueden
dispersar de forma elástica (Rayleigh) o inelástica (Raman).

En la primera forma los fotones absorbidos son emitidos de vuelta igual que
fueron absorbidos, la gran mayoría de ellos, pero una pequeña cantidad se emiten
cambiados, con una pequeña disminución o aumento de sus energía (ver
Figura~\ref{fig:dispersion-raman}).

Este cambio de energía varía según el material o elemento contra el que impacten
los fotones, revelando ahí la estructura o composición y abriendo un amplio
abanico de aplicaciones para esta
técnica\cite{what-is-raman,wiki:raman-scatter}. Con la dispersión Raman
capturada se obtienen los espectros Raman.

\imagen{dispersion-raman}{Representración de la dispersión de
	fotones\cite{what-is-raman}}

Con lo explicado anteriormente se puede ver que esta técnica tiene varias
ventajas, entre las que destacan\cite{what-is-raman}:
\begin{itemize}
	\tightlist
	\item Análisis sin contacto y no destructivo.
	\item No se suele necesitar preparar la muestra
	\item Sirve para materia orgánica e inorgánica.
	\item Se puede usar para elementos en cualquier estado de la materia
	\item Para conseguir un espectro la exposición de la muestra al láser está
	entre 10ms y 1s
\end{itemize}


\tablaSmall{Herramientas y tecnologías utilizadas en cada parte del proyecto}{l
	c c c c}{herramientasportipodeuso}
{ \multicolumn{1}{l}{Herramientas} & App AngularJS & API REST & BD & Memoria
	\\}{ 
	HTML5 & X & & &\\
	CSS3 & X & & &\\
	BOOTSTRAP & X & & &\\
	JavaScript & X & & &\\
	AngularJS & X & & &\\
	Bower & X & & &\\
	PHP & & X & &\\
	Karma + Jasmine & X & & &\\
	Slim framework & & X & &\\
	Idiorm & & X & &\\
	Composer & & X & &\\
	JSON & X & X & &\\
	PhpStorm & X & X & &\\
	MySQL & & & X &\\
	PhpMyAdmin & & & X &\\
	Git + BitBucket & X & X & X & X\\
	Mik\TeX{} & & & & X\\
	\TeX{}Maker & & & & X\\
	Astah & & & & X\\
	Balsamiq Mockups & X & & &\\
	VersionOne & X & X & X & X\\
} 

